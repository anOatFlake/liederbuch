\documentclass[../Liederbuch/LiederbuchGitarristen.tex]{subfiles}
\begin{document}
\song{Wingman}{Das Lumpenpack}{4}{Deutsch}{Liedermacher}{

\verse{
\li{\A[sus2]Ich muss euch jetzt mal was gestehen, \A[sus2]was euch sicher traurig stimmt.}
\li{\A[sus2]Doch ist es an der Zeit, \A[sus2]dass diese Lüge mal ein Ende nimmt.}
\li{\F[]Haltet euch gut fest, denn es \C[]zu sagen fällt mir schwer,}
\li{\G[]Batman, Flash und Spiderman sind \C[]alle\G[]samt \C[]nur \G[]imagi\A[sus2]när.}
}

\verse{
\li{\A[sus2]Ich weiß ihr seid geschockt, \A[sus2]und ich kann euch gut verstehen.}
\li{\A[sus2]Ohne Superhelden \A[sus2]würd die Welt doch untergehen.}
\li{\F[]Doch es gibt ihn ja den einen, den \C[]wahren Superheld.}
\li{\G[]Nacht für Nacht beschützt er uns und \C[]rettet \G[]diese \C[]Welt.\G}
}

\verse{
\li{\A[sus2]Man trifft ihn ziemlich oft, \A[sus2]doch erkennt ihn ziemlich selten.}
\li{\A[sus2]Denn er ist nicht der \A[sus2]Prototyp eines Superhelden.}
\li{\F[]Er kommt aus unsrer Mitte, er \C[]ist ein Freund von dir.}
\li{\G[]Die Liebe ist sein Anliegen, \C[]Verkuppeln \G[]sein \C[]Pläsier.\G* \A[sus2]}
}

\bridge{
\li{\F[]Die Frau sieht doch gut aus, du traust dich nicht ran?}
\li{\Em[]Da pirscht er schon vor und spricht sie gleich an.}
\li{\F[]Und hat sie 'ne nervige Freundin dabei, dann \Em[]knutscht er mit ihr}
\li{und schon ist der Weg \A[sus2]frei.}
}

\chorus{
\li{\F[]Ja das ist Wingman, der \C[]Retter in der \G[]Not.}
\li{\F[]Das ist Wingman, er bringt den \C[]Cockblockern den \G[]Tod.}
\li{\F[]Das ist Wingman, er löst \C[]jedes Pro\G[]blem.}
\li{\F[]Und hat sie ihren Freund dabei, dann \Em[]knutscht er halt mit dem!\A[sus2]}
}

\verse{
\li{\A[sus2]Die Frau dort in der Kneipe, \A[sus2]ist genau dein Typ.}
\li{\A[sus2]Schon direkt beim Reinkommen \A[sus2]hast du dich in sie verliebt.}
\li{\F[]Doch jetzt einfach rüber gehen ist ja \C[]auch nicht deine Art.}
\li{\G[]Erst mal wird die Frau noch zwei \C[]Stunden \G[]ange\C[]starrt.\G}
}

\verse{
\li{\A[sus2]Mit Bier trinkst du dir Mut an, \A[sus2]und bist schon bald beim Elften}
\li{\A[sus2]In dieser Situation kann \A[sus2]dir auch Superman nicht helfen.}
\li{\F[]Denn Superman kann fliegen und mit \C[]Laseraugen starren,}
\li{\G[]Wingmans Superkräfte sind sein \C[]Aussehen \G[]und \C[]sein \G[]Charme.}
}

\bridge{
\li{\F[]Ruf seinen Namen da kommt er schnurstracks \Em[]und spielt mit den Damen \dq Have you met Max?\dq}
\li{\F[]Und hat sie 'ne nervige Freundin dabei, \Em[]dann knutscht er mit ihr}
\li{und schon ist der Weg \A[sus2]frei.}
}

\refrain

\verse{
\li{\A[sus2]Wozu wäre ohne ihn – \A[sus2]diese Welt verkommen?}
\li{\A[sus2]Das Böse hätte längst schon \A[sus2]die Oberhand gewonnen.}
\li{\F[]Denn untervögelte Knaben, \C[]bergen Gefahr'n,}
\li{\G[]drum hilft Wingman, dass sie sich bei \C[]Zeiten \G[]auch \C[]'mal \G[]paar'n.}
}

\verse{
\li{\A[sus2]Und wenn ihr später fragt, \A[sus2]was ich denn von euch möcht':}
\li{\A[sus2]Das ist doch kein Superheld, \A[sus2]hat keine Superkräfte.}
\li{\F[]Dann kann ich euch nur sagen, \C[]darum gehts in diesem Lied.}
\li{\G[]Ein Superheld kann jeder sein, auch \C[]wenn man \G[]es nicht \C[]sieht.\G}
}

\bridge{
\li{\F[]Wir sind die Helden das musst du verstehen,}
\li{\Em[]jetzt fragst du natürlich \dq Wie soll das gehen?\dq}
\li{\F[]Wenn sich zu dir ein nerviges Mädchen gesellt,}
\li{\Em[]dann knutschst du mit ihr und rettest die \A[sus2]Welt.}
}

\chorus{
\li{\F[]Dann bist du Wingman, der \C[]Retter in der \G[]Not.}
\li{\F[]Dann bist du Wingman, du bringst den \C[]Cockblockern den \G[]Tod.}
\li{\F[]Dann bist du Wingman und löst \C[]jedes Pro\G[]blem.}
\li{\F[]Und hat sie ihren Freund dabei, dann \Em[]knutschst du halt auch mit dem!}
}

\refrain

\footer{
Cockblocker (aus dem Engl.) bezeichnet eine Person, die dich absichtlich oder unabsichtlich dabei stört einem potentiellen Partner Avancen zu machen. Eine anerkannte Methode sich von einem solchen Cockblocker zu befreien ist der sog. Wingman (aus dem Engl. Flügelmann): ein Freund oder eine Freundin die dich abschirmt, sodass du ungestört bleibst. Der Begriff stammt aus dem Militär und wurde durch den Roman \dq Väter und Söhne\dq\ (1862) erstmals umgewidmet. Besonders durch die Serie \dq How I Met Your Mother\dq\ wurde die Verwendung im heutigen Sprachgebrauch populär.}

}
\end{document}
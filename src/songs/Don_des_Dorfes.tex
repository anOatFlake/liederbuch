\documentclass[../Liederbuch/LiederbuchGitarristen.tex]{subfiles}
\begin{document}
\song{Don des Dorfes}{Das Lumpenpack}{3}{Deutsch}{Liedermacher}{

\intro{
\chli{\Am \Dm \Am \ \Am \Dm \Am \ \G \C \G \ \F \E[7]}
}

\verse{
\li{\Am[]Aufge\Dm[]wachsen zwischen \Am[]Feldern und Wald!}
\li{\Am[]Groß ge\Dm[]worden gleich den \Am[]Kälbern im Stall!}
\li{\G[]Wohl be\C[]hütet, heile \G[]Welt überall!}
\li{\F[]Fern von Sex, Drogen, Knast und \E[7]RTL II!}
}

\verse{
\li{\Am[]Hier geht's jeden \Dm[]morgen in die \Am[]Schule per Rad}
\li{und \Am[]einmal im \Dm[]Monat mit dem \Am[]Zug in die Stadt.}
\li{Für das \G[]Schnaps kaufen \C[]steht der große \G[]Bruder parat.}
\li{\F[]Hier wo man noch T-Shirts aus \E[7]Kuhleder hat!}
}

\verse{
\li{Ja, ich \Am[]hab zwar ein \Dm[]Handy, aber \Am[]niemals Empfang.}
\li{Das \Am[]ist auch nicht so \Dm[]wichtig nur deine \Am[]Mutter ruft dich an.}
\li{Zwei \G[]Dosen mit 'ner \C[]Kordel zu den \G[]Nachbarn gespannt,}
\li{denn \F[]meine besten Freunde wohn' di\E[7]rekt nebenan!}
}

\chorus{
\li{\F[]Ich bin der \G[]Don dieses \Am[]Dorfes! \G[]}
\li{\F[]Ich bin der \G[]Kaiser des \Am[]Kaffs! \Dm[]}
\li{\F[]Und ich \G[]muss hier nicht \Am[]fort, denn ich \G[]hab es ge\F[]schafft:}
\li{Einen \G[]Stammplatz in der \Am[]Kneipe, wo ich \G[]anschreiben \F[]kann,}
\li{Mein \G[]Bäcker fragt \dq Wie \Am[]immer?\dq , mein Fri\G[]seur kennt meinen \F[]Nam'n!}
\li{Und \G[]nichts auf der \Am[]Welt fehlt mir \G[]zu meinem \F[]Glück!}
}

\bridge{
\li{Auf der Straße grüß' ich \E[7]Omas – und sie grüßen zu\Am[]rück!}
\chli{\Am \Dm \Am \ \Am \Dm \Am \ \G \C \G \ \F \E[7]}
}

\verse{
\li{Du \Am[]pennst bis um \Dm[]vier und tanzt dann \Am[]bis in die Puppen!}
\li{Mir reichts zum \Am[]Feierabend\Dm[]bier 'nen AR\Am[]D-Film zu gucken.}
\li{Denn wenn der \G[]Hahn morgens \C[]kräht, dann ist der \G[]Schlaf nicht von Dauer!}
\li{Mit beiden \F[]Händen voll zu tun: \dq Schaffe, \E[7]schaffe, Häusle baue!\dq}
}

\verse{
\li{Mein \Am[]Kleidungsstil ist \Dm[]retro, das \Am[]steht außer Frage,}
\li{weil ich seit \Am[]Jahren die Kla\Dm[]motten der Geschwis\Am[]ter auftrage!}
\li{Ich \G[]hab' zwar nur Schwes\C[]tern, doch enge \G[]Hosen sind Trend!}
\li{Die \F[]Marken sind egal, Hauptsach' der Trek\E[7]ker ist von Fendt!}
}

\refrain

\bridge{
\li{Ich hup' 'nem Girl aus mein'm \E[7]Fendt zu, und sie rennt nicht gleich \Am[]weg!}
\chli{\Am \Dm \Am \ \Am \Dm \Am \ \G \C \G \ \F \E[7]}
\li{(Jetzt renn doch nicht! Renn doch nicht kleines Mädchen! Renn doch nicht! Ich hab' Süßes!)}
}

\verse{
\li{Die \Am[]Welt ist hier noch \Dm[]einfach, der Hori\Am[]zont ist sehr klein!}
\li{Der \Am[]Bus hält hier nur \Dm[]Sonntags, aber \Am[]niemals steigt wer ein!}
\li{Denn wir \G[]sind doch alle glück\C[]lich zwischen \G[]Gerste und Mais,}
\li{Vor\F[]ausgesetzt man ist katholisch, \E[7]hetero und weiß!}
}

\outro{
\li{\F[]Ich bin der \G[]Don dieses \Am[]Dorfes! \G[]}
\li{\F[]Ich bin der \G[]Kaiser des \Am[]Kaffs! \Dm[]}
\li{\F[]Und ich \G[]will hier nicht \Am[]fort, denn ich \G[]hab's nicht ge\F[]rafft:}
\li{Einen \G[]Stammplatz in der \Am[]Kneipe, wo ich \G[]anschreiben \F[]kann,}
\li{ein \G[]Bäcker der was \Am[]fragt, ein Fri\G[]seur mit meinem \F[]Nam'n!}
\li{Vielleicht \G[]fehlt mir die \Am[]Welt, \G[]zu meinem \F[]Glück,}
\li{doch wer soll's mir er\E[7]zählen? …es kam keiner zu\Am[]rück!}
}

}
\end{document}
\documentclass[../Liederbuch/LiederbuchGitarristen.tex]{subfiles} 
\begin{document}
\song{Streets of London}{Ralph McTell}{4}{Englisch}{Folk}{
\verse{
\li{\C[]Have you seen the \G[]old man in the \Am[]closed down \Em[]market,}
\li{\F[]kicking up the \C[]papers with his \Dm[]worn out \G[]shoes?}
\li{\C[]In his eyes you \G[]see no pride, \Am[]hand held loosely \Em[]by his side,}
\li{\F[]yesterday's \C[]paper telling \G[]yesterday's \C[]news.}
}

\chorus{
\li{So \F[]how can you \Em[]tell me you're\ch{(} \C[]lo\Em[)]ne-\Am[]ly,}
\li{\D[]and you say for you that the sun don't \G[]shine?\G[7]}
\li{\C[]Let me take you \G[]by the hand and \Am[]lead you through the \Em[]streets of London,}
\li{\F[]I'll show you \C[]something that'll \G[]make you change your \C[]mind.}
}

\verse{
\li{\C[]Have you seen the \G[]old girl who \Am[]walks the streets of \Em[]London?}
\li{\F[]Dirt in her \C[]hair and her \Dm[]clothes in \G[]rags.}
\li{\C[]She's no time for \G[]talking, she \Am[]just keeps right on \Em[]walking,}
\li{\F[]carrying her \C[]home in two \G[]carrier \C[]bags.}
}

\refrain

\verse{
\li{\C[]In the all night \G[]café at a \Am[]quarter past e\Em[]leven,}
\li{\F[]same old \C[]man sitting \Dm[]there on his \G[]own.}
\li{\C[]Looking at the \G[]world over the \Am[]rim of his \Em[]tea cup,}
\li{\F[]each tea lasts an \C[]hour, then he \G[]wanders home a\C[]lone.}
}

\refrain

\verse{
\li{\C[]Have you seen the \G[]old man out\Am[]side the seamen's \Em[]mission?}
\li{\F[]Memory fading \C[]with the medal \Dm[]ribbons that he \G[]wears.}
\li{\C[]In this lonesome \G[]city the rain \Am[]cries a little \Em[]pity,}
\li{for \F[]one more forgotten \C[]hero and a \G[]world that doesn't \C[]care.}
}

\refrain
\footer{Das Lied ist von McTells Erfahrungen beim Trampen durch Europa geprägt,
besonders von Paris. Urspünglich sollte das Lied daher auch \dq Streets of Paris\dq{} heißen.}
}
\end{document}
\documentclass[../Liederbuch/LiederbuchGitarristen.tex]{subfiles} 
\begin{document}
\song{Badnerlied}{trad.}{}{Deutsch}{}{
\verse{
\li{Das \G[]schönste Land in Deutschlands Gau'n, das \D[]ist mein Badner \G[]Land.}
\li{Es \C[]ist so herrlich \D[]anzuschau'n und \A[7]ruht in Gottes \D[]Hand.}
}

\chorus{
\li{Drum grüß ich dich mein Badner\D[7]land!}
\li{Du \G[]edle Perl’ im deutschen \D[]Land.}
\li{Frischauf, frischauf, frischauf, frischauf,}
\li{frischauf, frischauf mein \A[7]Badner\D[]land.}
}

\verse{
\li{Zu \G[]Haslach gräbt man Silbererz,
bei \D[]Freiburg wächst der \G[]Wein,}
\li{im \C[]Schwarzwald schöne \D[]Mädchen,
ein \A[7]Badner möcht’ ich \D[]sein.}
}

\refrain

\verse{
\li{Zu \G[]Karlsruh' ist die Residenz, in \D[]Mannheim die Fa\G[]brik,}
\li{in \C[]Rastatt ist die \D[]Festung, und \A[7]das ist Badens \D[]Glück! }
}

\refrain

\verse{
\li{Alt-\G[]Heidelberg, du feine, du \D[]Stadt an Ehren \G[]reich.}
\li{Am \C[]Neckar und am \D[]Rheine, keine \A[7]and're kommt dir \D[]gleich. }
}

\refrain

\verse{
\li{Der \G[]Bauer und der Edelmann, das \D[]stolze Mili\G[]tär,}
\li{die \C[]schau'n einander \D[]freundlich an, und \A[7]das ist Badens \D[]Ehr'.}
}

\refrain

}
\end{document}
\documentclass[../Liederbuch/LiederbuchGitarristen.tex]{subfiles}
\begin{document}
\song{Einheitsfrontlied}{T:Bertolt Brecht M:Hanns Eisler}{}{Deutsch}{}{

\verse{
\li{Und \Em weil der Mensch ein \B[7]Mensch ist,}
\li{drum braucht er was zum Essen bitte \Em sehr!}
\li{Es \E[7]macht ihn kein Ge\Am[]schwätz nicht satt,}
\li{das \B[7]schafft Essen \Em her!}
}

\chorus{
\li{Drum \Em links, zwei, drei, drum \B[7]links, zwei, drei,}
\li{wo dein \Em Platz Genosse \Am ist.}
\li{Reih dich\ch{(} \C ein in die \D[7)]Arbeite\ch{(}r\G[]einheits\Em[)]front,}
\li{weil du \B[7]auch ein Arbeiter \Em bist.}
}

\verse{
\li{Und \Em weil der Mensch ein \B[7]Mensch ist,}
\li{drum braucht er auch noch Kleider und \Em Schuh.}
\li{Es \E[7]macht ihn kein Ge\Am[]schwätz nicht warm}
\li{und \B[7]auch kein Trommeln da\Em[]zu!}
}

\refrain

\verse{
\li{Und \Em weil der Mensch ein \B[7]Mensch ist,}
\li{drum hat er Stiefel im Gesicht nicht \Em gern.}
\li{Er \E[7]will unter sich keinen \Am Sklaven sehen}
\li{und \B[7]über sich keinen \Em Herren.}
}

\refrain

\verse{
\li{Und \Em weil der Prolet ein Pro\B[7]let ist,}
\li{drum kann ihn auch kein anderer be\Em[]frei'n.}
\li{Es \E[7]kann die Befreiung der \Am Arbeiter nur}
\li{das \B[7]Werk der Arbeiter \Em sein.}
}

\refrain

\footer{Das Einheitsfrontlied entstand 1934 in dem Versuch, die zerstrittenen Anhänger von SPD und KPD angesichts der Machtergreifung der Nationalsozialisten wieder zu vereinigen. Bis heute gehört es zu den bekanntesten deutschen Arbeiterliedern.}
}
\end{document}
\documentclass[../Liederbuch/LiederbuchGitarristen.tex]{subfiles} 
\begin{document}
\song{Der Bauch des Spielmanns}{Die Galgenvögel}{}{Deutsch}{Mittelalterlich}{
\intro{
\chli{\brep\Em \C \G \D \Em \G \D \Em\erep {} \Em}
}

\verse{
\li{\Em Wenn wir in der \D Schänke hängen \Em und uns nach dem \B[7]Biere drängen,}
\li{\C wenn wir uns're \G Lieder singen \C und dazu die \D Saiten klingen,}
\li{so \Em bringen \C wir nach \G alter \D Weise \Em unser \C Prosit \B[7]auf die Reise}
}

\chorus{
\li{\Em Mägdelein \C reib mir noch\G[]mal übern \D Bauch!}
\li{\Em Zier dich nicht \C so, du \D willst es doch \Em auch.}
\li{Ach ja, \Em Mägdelein \C reib mir noch\G[]mal übern \D Bauch!}
\li{\Em Zier dich nicht \C so, du \D willst es doch \Em auch.\ch{(} \Em* \C* \G* \D* \Em* \G* \D* \Em[) \rep{2}]* \Em}
}


\verse{
\li{\Em Alle, die vom \D Suff getrieben, \Em schnorrend durch die \B[7]Lande ziehen,}
\li{\C jene, die dem \G Weine trotzen, der \C Schankmaid in den \D Ausschnitt kotzen.}
\li{Ja, \Em diesen \C Burschen \G gilt die \D Stunde \Em doch den \C Spießern \B[7]diese Kunde.}
}

\refrain

\verse{
\li{\Em Halvdan muss die \D Zeche zahlen, \Em seht ihn nur in \B[7]seinen Qualen.}
\li{\C Sehnt sich nach dem \G vollen Becher, \C ist er doch der \D schlimmste Zecher.}
\li{Oh, \Em Freunde \C lasst uns \G nicht ver\D[]zagen, \Em und den \C Wirt zum \B[7]Teufel jagen.}
}

\refrain

\verse{
\li{\Em Sitzen wir zu \D später Stunde, \Em schlucken unsre \B[7]letzte Runde,}
\li{\C alles durchein\G[]ander saufen, \C musst du auch zum \D Abtritt laufen.}
\li{\dq\Em Bruder, \C mach uns \G keine \D Schande!\dq, \Em grölt die \C ganze \B[7]Säuferbande.}
}

\refrain

\verse{
\li{\Em Soll der Gersten\D[]saft uns munden, \Em Galgenvögel, \B[7]schräge Kunden.}
\li{\C Schwätzer und Ta\G[]vernenspinner, \C Saufen, Huren das \D prägt für immer.}
\li{Erst \Em wenn wir \C unterm \G Tische \D liegen \Em grinst die \C Schankmaid \B[7]stets zufrieden.}
}

\outro{
\li{\Em Mägdelein \C reib mir noch\G[]mal übern \D Bauch!}
\li{\Em Zier dich nicht \C so, du \D willst es doch \Em auch.}
\li{Ach ja, \Em Mägdelein \C reib mir noch\G[]mal übern \D Bauch!}
\li{\Em Zier dich nicht \C so, du \D willst es doch \Em auch.}
\li{\Em Mägdelein \C reib mir noch\G[]mal übern \D Bauch!}
\li{\Em Zier dich nicht \C so, du \D willst es doch \Em auch.}
\li{Denn \Em wisset, ja \C ihn zu \G reiben bringt \D Glück. des \Em Spielmanns \C Bauch, sein \D bestes \Em Stück.}
}

}
\end{document}
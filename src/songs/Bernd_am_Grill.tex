\documentclass[../Liederbuch/LiederbuchGitarristen.tex]{subfiles} 

\begin{document}
\song{Bernd am Grill}{Hasenscheisse}{}{Deutsch}{Liedermacher}{
\chorus{
\li{Ick hab die \C Schürze um, ick dreh die \F Würschte um, ick bin der \C Bernd und steh am \G Grill.}
\li{Und mit'm \C Bierchen in der Kralle grill ick - \F immer wird et alle.}
\li{Fleisch und\ch{(} \C Würschte grill'n is \G[)]allet wat ick \C will.}
}

\refrain

\verse{
\li{Und leckere \C Wurscht hab ick wie wild, beim letzten \F Dorffest erst gegrillt,}
\li{da \C kam sogar aus'm Nachbarort Be\G[]such.}
\li{Ick grillte \C viel, ick grillte flott, ick grillte \F wie'n junger Gott,}
\li{nur b\ch{(}e\C[]waffnet mit 'ner \G[)]Jabel und 'nem \C Tuch.}
\li{Doch dann \C kam zur Party eene, Krügers \F Tochter, ja die kleene,}
\li{in die \C hatt' ick ma vor Jahren schon ver\G[]guckt.}
\li{Ick knabber \C grad in Ruh' 'ne Wiener, kam die \F of ma zu, die Diva,}
\li{hatt ick\ch{(} \C prustend ma am \G[)]Ende fast ver\C[]schluckt.}
}

\refrain

\verse{
\li{Ick denke, \dq\C Scheiße, die will tanzen!\dq, tu mich schon \F hinterm Grill verschanzen,}
\li{als da \C plötztlich vor mir noch wer andret \G steht:}
\li{Det war \C unser Bürgermeister, ziemlich \F fett und Schulze heißt er,}
\li{sacht nur\ch{(} \C trocken \dq Tach\dq {} und \G[)]fragt ma wie't so \C jeht.}
\li{Ick sach, \dq Na \C muss ja, wa, du Sau? Was macht der \F Hof, was macht die Frau?\dq,}
\li{doch den \C Schulze bringt det jar nich aus der \G Ruh'.}
\li{Der sacht, \dq Ne, \C Bernd, jetzt ma in echt, du grillst so \F gut, ick grill so schlecht.}
\li{Wat em\ch{(}p\C[]fiehlst du und wie \G[)]grillt man denn wie \C du?\dq}
}

\refrain

\verse{
\li{Naja, und dann hab ick ihm erst ma Bescheid gestoßen,}
\li{hab den Grill gefettet, die Schürze festgezurrt,}
\li{und meinte: \dq Na bei den \C Steaks, da musste kieken, darfste \F nich so viel rumpieken,}
\li{sieht doch \C doof aus, außerdem verlier'n se \G Saft.}
\li{Hier vorne \C links die Kräuterwurscht schmeckt ziemlich \F jut, macht aber Durscht,}
\li{ick em\ch{(}p\C[]fehl dir 'ne Bu\G[)]lette, det jibt \C Kraft!\dq}
\li{Doch er sacht: \dq \C Ne, weeste wat? Gib mir 'ne \F Currywurscht, macht satt!\dq}
\li{Ick geb ihm \C hin det trockne Ding und sach: \dq Bis \G bald!\dq}
\li{Inzwischen \C stand die holde Maid vom Grill nur \F zwanzig Meter weit,}
\li{sie is eins\ch{(} \C siebzig groß und \G[)]zwanzig Jahre \C alt.}
}

\refrain

\verse{
\li{Ick stand noch \C lässig an de Wand, mit meener \F Zange in de Hand,}
\li{und frachte \C cool: \dq Wat willste, schönet \G Kind?}
\li{Sie sacht, sie \C will keene Bulette und die \F Schnitzel wär'n zu fette,}
\li{sie will 'ne\ch{(} \C Currywurscht, weil \G[)]die so lecker \C sind.}
\li{Und ick \C kucke uff'n Teller, scheiße, \F Schulze der war schneller,}
\li{hat sich \C glatt die letzte Currywurscht ge\G[]krallt.}
\li{Ich schrei ihm \C nach, \dq Det war nich fair, komm jib die \F Wurscht mir wieder her!\dq,}
\li{auf hunder\ch{(}t\C[]achtzig und die \G[)]Hand zur Faust ge\C[]ballt.}
}

\refrain

\verse{
\li{Doch bald schon \C macht sich mit der Zeit, Unruhe in \F meinem Körper breit,}
\li{nur een Je\C[]danke: \dq Wie krieg ick det Mädel \G satt?\dq}
\li{Ick fragte \C Schulze wat nu is, \dq Gib ihr doch \F wenigstens 'nen Biss!\dq,}
\li{aber der\ch{(} \C rief, \dq Sach ma, \G[)]spinnste, oder \C wat?\dq}
\li{Verzweifelt \C wühlte ick im Dreck, die kleene \F Krüger war längst weg,}
\li{ick \C suchte nach der Wurst zum großen \G Glück.}
\li{Ich tanzte \C wütend rum im Kreise, \dq Schulze, \F du hast doch 'ne Meise!\dq}
\li{War det zu\ch{(} \C glauben, war die \G[)]janze Welt ver\C[]rückt?}
\li{Doch plötzlich \C aufgebrachte Leute und eener \F ruft, \dq Bernd, schläfst du heute?\dq,}
\li{und ick be\C[]merkte een Jeruch und det Je\G[]kreisch}
\li{stach in der \C Nase mir wie Kot. \dq Bernd, det is \F schlimmer als der Tod!\dq}
\li{\ch{(}F\C[]reunde, ja ick \G[)]roch verbranntet \C Fleisch!}
}

\refrain

\verse{
\li{Und ich sprang \C uff, ganz aus'm Häuschen, die lachten \F laut und sich ins Fäustchen,}
\li{ach wat ein \C Elend, da war allet uff'm \G Grill}
\li{raben\C[]schwarz, nich mehr zu retten, die juten \F Steaks, die schönen Buletten,}
\li{det bleibt ve\ch{(}r\C[]brannt, da kann man \G[)]machen, was man \C will.}
\li{Die Wurscht war \C nur noch schwarze Strippe, die \F Rippchen nur noch so Jerippe,}
\li{meine \C Ehre war dahin, ick grill nie \G wieder.}
\li{Det is det \C Schlimmste von der Welt, noch schlimmer wie wenn's \F Schlachtefest ausfällt,}
\li{und voller\ch{(} \C Demut legt ick \G[)]meine Schürze \C nieder.}
\li{Und schließlich \C kriech ick unter'n Grill, um mich he\F[]rum wird allet still,}
\li{und ich er\C[]inner mich, wie ick als kleener \G Bu,}
\li{in meene \C erste Wurscht reinbeiße, frische \F Schlachtewurscht, janz heiße,}
\li{und dann\ch{(} \C schrei ick's raus mit \G[)]neu jewonnenem \C Mut:}
}

\refrain

\verse{
\li{Ei, beim \C Grillen, da tu ick streben, da lass ick \F mir von keinem rinreden,}
\li{ja vor \C mir zieht selbst der Schulze seinen \G Hut.}
\li{Damals, \C in der Fleischerlehre, kam mir \F keiner in die Quere,}
\li{und wenn\ch{(} \C doch, dann jab et \G[)]Saures und viel \C Blut!}
\li{Na, ick ver\C[]kaufe jetzt Karotten, \F Vollkornbrot und Haferflocken.}
\li{\C Fleisch sollte meine Sache nicht mehr \G sein.}
\li{Man fragt mich, \dq\C Bernd is allet klar?\dq, und ick sage, \dq\F Na muss ja, wa?\dq,}
\li{denn ick wees,\ch{(} \C eines Tages \G[)]werd ick's wieder \C schreien:}
}

\repref{2}

}
\end{document}
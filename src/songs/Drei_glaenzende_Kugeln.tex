\documentclass[../Liederbuch/LiederbuchGitarristen.tex]{subfiles} 
\begin{document}
\song{Drei glänzende Kugeln}{Franz Josef Degenhardt}{}{Deutsch}{}{
\verse{
\li{Es \Am[]liegen drei \E[]glänzende \Am[]Kugeln, ich \Dm[]weiß nicht, wo\E[]raus ge\Am[]macht,}
\li{in \Am[]einer \E[]niedrigen \Am[]Kneipe neun \Dm[]Meilen hin\E[]ter der \Am[]Nacht.}
\li{Sie \E[]liegen auf grünem \Am[]Tuch und \E[]an der Wand steht der \E[7]Spruch:}
}

\chorus{
\li{\F[]Wer die Kugeln \C[]rollen lässt, \Dm[]dadi \G[]daredi \C[]dum dei}
\li{\F[]den überkomme die \C[]schwarze Pest \E[]daredi \E[7]daredi \Am[]dum}
}

\verse{
\li{Der \Am[]Wirt der \E[]hat nur ein \Am[]Auge und \Dm[]das trägt er \E[]hinter dem \Am[]Ohr.}
\li{Aus \Am[]seinem ge\E[]spaltenen \Am[]Kopfe ragt \Dm[]eine An\E[]tenne her\Am[]vor.}
\li{Er \E[]trinkt aus reiner \Am[]Seele und \E[]ruft aus roter \E[7]Kehle:}
}

\refrain

\verse{
\li{Die \Am[]einen \E[]sagen, die \Am[]Kugeln sind die \Dm[]Sonne, die \E[]Erde, der \Am[]Mond.}
\li{Die \Am[]andern \E[]glauben, sie \Am[]seien das \Dm[]Feuer, die \E[]Angst und der \Am[]Tod.}
\li{Und \E[]wenn sie beisammen \Am[]sind, dann \E[]raunen sie in den \E[7]Wind:}
}

\refrain

\verse{
\li{Doch \Am[]dann kam \E[]einer ge\Am[]ritten - es \Dm[]war in dem \E[]Jahr vor der \Am[]Zeit -}
\li{auf \Am[]einer ge\E[]sattelten \Am[]Wolke von \Dm[]hinter der \E[]Ewig\Am[]keit.}
\li{Er \E[]nahm von der Wand einen \Am[]Queue, der \E[]Wirt rief krächzend: \dq\E[7]Hey!\dq}
}

\refrain

\verse{
\li{Doch \Am[]jener \E[]lachte zwei \Am[]Donner und \Dm[]wachste den \E[]knöchernen \Am[]Stab}
\li{vi\Am[]sierte, \E[]stieß und die \Am[]Kugeln \Dm[]prallten, der \E[]Wirt grub ein \Am[]Grab.}
\li{\E[]Fäulnis flatterte \Am[]auf, so \E[]nahm alles seinen \E[7]Lauf.}
}

\refrain
\footer{Das Lied thematisiert die Erforschung und Gefahren der Atomkraft. Die „9 Meilen“ sind eine Anspielung auf die Ortschaft Nine Miles, ein Nest vor den Toren von Los Alamos/New Mexico, das einzig für die Atombombenforschung errichtet wurde und heute nicht mehr existiert. Der Billard-Vergleich stammt von Otto Hahn.}
}
\end{document}
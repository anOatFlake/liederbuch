\documentclass[../Liederbuch/LiederbuchGitarristen.tex]{subfiles}
\begin{document}
\song{The Foggy Dew}{Charles O'Neill}{2}{Englisch}{Irisch/Schottisch}{

\verse{
\li{As \Am[]down the glen one \G[]Easter morn' to a \C[]city \G[]fair rode \Am[]I,}
\li{there armed lines of \G[]marching men in \C[]squadrons \G[]passed me \Am[]by.}
\li{No \C[]pipe did hum, nor \G[]battle \Am[]drum did sound it's \Em[]loud ta\Am{}ttoo,}
\li{but the Angelus bell o'er the \G[]Liffey's swell rang \C[]out in the \F[]Foggy \Am[]Dew.}
}

\verse{
\li{Right \Am[]proudly high o'er \G[]Dublin Town they \C[]hung out the \G[]flag of \Am[]war,}
\li{'t'was better to die 'neath an \G[]Irish sky than at \C[]Suvla or \G[]Sud El \Am[]Bar.}
\li{And \C[]from the plains of \G[]Royal \Am[]Meath strong men came \Em[]hurrying \Am[]through,}
\li{while Brittania's huns, with their \G[]long range guns, sailed \C[]in through the \F[]Foggy \Am[]Dew.}
}

\verse{
\li{Oh, the \Am[]night fell black, and the \G[]rifles' crack made \C[]perfidious \G[]Albion \Am[]reel,}
\li{'mid the leaden rain, seven \G[]tongues of flame did \C[]shine o'er the \G[]lines of \Am[]steel.}
\li{By each \C[]shinning blade a \G[]prayer was \Am[]said that to Ireland her \Em[]sons be \Am[]true,}
\li{but when morning broke still the \G[]war flag shook out it's \C[]folds in the \F[]Foggy \Am[]Dew.}
}

\verse{
\li{'T'was \Am[]England bade our \G[]Wild Geese go that \C[]small nations \G[]might be \Am[]free,}
\li{but their lonely graves are by \G[]Suvla's waves or the \C[]fringe of the \G[]Great North \Am[]Sea.}
\li{Oh \C[]had they died by \G[]Pearse's \Am[]side or had fought with \Em[]Cathal \Am[]Brugha,}
\li{their names we'd keep where the \G[]Fenians sleep, 'neath the \C[]shroud of the \F[] Foggy \Am[]Dew.}
}

\verse{
\li{But the \Am[]bravest fell, and the \G[]requiem bell rang \C[]mournful\G[]ly and \Am[]clear,}
\li{for those who died that \G[]Eastertide in the \C[]springtime \G[]of the \Am[]year.}
\li{When the \C[]world did gaze with \G[]deep a\Am[]maze on those fearless \Em[]men but \Am[]few,}
\li{who bore the fight so that \G[]freedom's light might \C[]shine through the \F[]Foggy \Am[]Dew.}
}

\verse{
\li{Back \Am[]through the glen I \G[]rode again, and my \C[]heart with \G[]grief was \Am[]sore,}
\li{for I parted then with \G[]valient men who I \C[]never shall \G[]see no \Am[]more.}
\li{But \C[]to and fro' in my \G[]dreams I \Am[]go, and I kneel and I \Em[]pray for \Am[]you,}
\li{for slavery fled, oh \G[]glorious dead, when you \C[]fell in the \F[]Foggy \Am[]Dew.}
}

\footer{In dem Lied verarbeitet der Priester Charles O'Neill die Ereignisse des irischen Osteraufstands 1916. Zunächst unterstützen die meisten Iren Englands Kampf im Ersten Weltkrieg. Als jedoch 1915 viele Iren in der Schlacht von Gallipoli ('Suvla or Sud El Bar') fielen, wurde die Forderung laut, dass Englands Begründung für den Krieg, die Freiheit kleinerer Länder, auch für Irland gelten müsse. An Ostermontag 1916 besetzten dann Milizen Teile Dublins. Die harte Niederschlagung des Aufstands, bei der die Engländer auch Zivilisten erschossen und Dublin niederbrannte sowie die Hinrichtung der Anführer führten zur verbreiteter antibritischer Stimmung in Irland, die in den Irischen Unabhängigkeitskrieg 1919-1921 mündete.}
}
\end{document}
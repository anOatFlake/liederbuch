\documentclass[../Liederbuch/LiederbuchGitarristen.tex]{subfiles} 
\begin{document}
\song{Moorsoldaten}{T:J. Esser/W. Langhoff M:R. Goguel}{}{Deutsch}{Liedermacher}{
\verse{
\li{\Em[]Wohin auch das Auge blicket: \Am[]Moor und \Em[]Heide \B[7]nur rings\Em[]um.}
\li{\G[]Vogelsang uns nicht erquicket, \Am[]Eichen \Em[]stehen \B[7]kahl und \Em[]krumm.}
}

\chorus{
\li{\brep Wir \G[]sind die Moorsol\D[]daten und \Em[]ziehen mit dem \B[7]Spaten ins \Em[]Moor.\erep}
}

\verse{
\li{\Em[]Hier in dieser öden Heide \Am[]ist das \Em[]Lager \B[7]aufge\Em[]baut,}
\li{\G[]wo wir fern von jeder Freude \Am[]hinter \Em[]Stachel\B[7]draht ver\Em[]staut.}
}

\refrain

\verse{
\li{\Em[]Morgens ziehen die Kolonnen \Am[]durch das \Em[]Moor zur \B[7]Arbeit \Em[]hin.}
\li{\G[]Graben bei dem Brand der Sonne, \Am[]doch zur \Em[]Heimat \B[7]steht der \Em[]Sinn.}
}

\refrain

\verse{
\li{\Em[]Heimwärts, heimwärts, jeder sehnet \Am[]zu den \Em[]Eltern, \B[7]Weib und \Em[]Kind.}
\li{\G[]Manche Brust ein Seufzer dehnet, \Am[]weil wir \Em[]hier ge\B[7]fangen \Em[]sind.}
}

\refrain

\verse{
\li{\Em[]Auf und nieder geh'n die Posten, \Am[]keiner, \Em[]keiner \B[7]kann hin\Em[]durch.}
\li{\G[]Flucht wird nur das Leben kosten, \Am[]vierfach \Em[]ist um\B[7]zäunt die \Em[]Burg.}
}

\refrain

\verse{
\li{\Em[]Doch für uns gibt es kein Klagen, \Am[]ewig \Em[]kann's nicht \B[7]Winter \Em[]sein.}
\li{\G[]Einmal werden froh wir sagen: \Am[]Heimat, \Em[]du bist \B[7]wieder \Em[]mein.}
}

\chorus{
\li{\brep Dann \G[]zieh'n die Moorsol\D[]daten \Em[]nicht mehr mit dem \B[7]Spaten ins \Em[]Moor!\erep}
}
\footer{Das Moorsoldatenlied wurde 1933 von Häftlingen des Konzentrationslagers Börgermoor bei Papenburg im Emsland geschaffen. In diesem Lager wurden vorwiegend politische Gegner des Nazi-Regimes gefangen gehalten. Mit einfachen Werkzeugen wie dem Spaten mussten diese dort das Moor kultivieren.}
}
\end{document}
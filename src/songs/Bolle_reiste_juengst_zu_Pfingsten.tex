\documentclass[../Liederbuch/LiederbuchGitarristen.tex]{subfiles} 
\begin{document}
\song{Bolle reiste jüngst zu Pfingsten}{trad.}{}{Deutsch}{}{
\verse{
\li{Bolle \G[]reiste jüngst zu \C[]Pfingsten, nach \D[7]Pankow war sein \G[]Ziel,}
\li{da ver\G[]lor er seinen \C[]Jüngsten ganz \D[7]plötzlich im Je\G[]wühl.}
\li{'Ne \A[]volle halbe Stunde hat \A[7]er nach ihm ge\D[]spürt,}
\li{\brep\D[7]aber \G[]dennoch hat sich \C[]Bolle janz \D[7]köstlich amü\G[]siert!\erep}
}

\verse{
\li{Zu \G[]Pankow gab's kein \C[]Essen, zu \D[7]Pankow gab's kein \G[]Bier}
\li{war \G[]alles aufge\C[]fressen von \D[7]fremden Leuten \G[]hier.}
\li{Nicht \A[]mal 'ne Butterstulle hat \A[7]man ihm reser\D[]viert,}
\li{\brep\D[7]aber \G[]dennoch hat sich \C[]Bolle ganz \D[7]köstlich amü\G[]siert!\erep}
}

\verse{
\li{Auf der \G[]Schöneholzer \C[]Heide, da \D[7]gab's 'ne Keile\G[]rei.}
\li{Und \G[]Bolle, gar nicht \C[]feige, war \D[7]mittendrin da\G[]bei.}
\li{Hat's \A[]Messer rausgerissen und \A[7]Fünfe massa\D[]kriert,}
\li{\brep\D[7]aber \G[]dennoch hat sich \C[]Bolle ganz \D[7]köstlich amü\G[]siert!\erep}
}

\verse{
\li{Es \G[]fing schon an zu \C[]tagen, als \D[7]er sein Heim er\G[]blickt.}
\li{Das \G[]Hemd war ohne \C[]Kragen, das \D[7]Nasenbein zer\G[]knickt,}
\li{das \A[]linke Auge fehlte, das \A[7]Rechte marmo\D[]riert,}
\li{\brep\D[7]aber \G[]dennoch hat sich \C[]Bolle ganz \D[7]köstlich amü\G[]siert!\erep}
}

\verse{
\li{Als \G[]er nach Haus ge\C[]kommen, da \D[7]ging's ihm aber \G[]schlecht.}
\li{Da \G[]hat ihn seine \C[]Olle ganz \D[7]mörderisch ver\G[]drescht!}
\li{'Ne \A[]volle halbe Stunde hat \A[7]sie auf ihm po\D[]liert,}
\li{\brep\D[7]aber \G[]dennoch hat sich \C[]Bolle ganz \D[7]köstlich amü\G[]siert!\erep}
}
\footer{'Bolle reiste jüngst zu Pfingsten' ist eines der bekanntesten Volkslieder aus dem Berliner Raum im Berlinischen Dialekt. Der verbreitete Spitzname Bolle (Zwiebel) steht für eine nicht näher bestimmte Person.}
}
\end{document}
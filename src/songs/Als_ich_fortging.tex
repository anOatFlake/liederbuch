\documentclass[../Liederbuch/Liederbuch.tex]{subfiles} 
\begin{document}
\song{Als ich fortging}{Karussell}{}{Deutsch}{}{

\verse{
\li{\Em[]Als ich fortging war die \Am[]Straße steil, \D[]kehr wieder \G[]um!}
\li{\C[]Nimm an ihrem Kummer \D[]teil, mach sie \Em[]heil.}
\li{Als ich fortging war der \Am[]Asphalt heiß, \D[]kehr wieder \G[]um!}
\li{\C[]Red ihr aus um jeden \D[]Preis, was sie \Em[]weiß.}
}

\chorus{
\li{\C[]Nichts ist un\G[]endlich, so \Am[]sieh das doch \Em[]ein. }
\li{Ich \C[]weiß du willst unendlich \D[]sein, schwach und \Em[]klein. }
\li{\C[]Feuer brennt \G[]nieder, wenn's \Am[]keiner mehr \Em[]nährt. }
\li{\C[]Kenn ja selber, was dir \D[]heut wider\Em[]fährt.}
}

\verse{
\li{\Em[]Als ich fortging war'n die \Am[]Arme leer, \D[]kehr wieder \G[]um!}
\li{\C[]Mach sie leichter einmal \D[]mehr, nicht so \Em[]schwer. }
\li{Als ich fortging kam ein \Am[]Wind so wach, \D[]warf mich nicht \G[]um. }
\li{\C[]Unter ihrem Tränen\D[]dach war ich \Em[]schwach.}
}

\chorus{
\li{\C[]Nichts ist un\G[]endlich, so \Am[]sieh das doch \Em[]ein. }
\li{Ich \C[]weiß du willst unendlich \D[]sein, schwach und \Em[]klein. }
\li{\C[]Nichts ist von \G[]Dauer, was \Am[]keiner recht \Em[]will. }
\li{\C[]Auch die Trauer wird dann \D[]sein, schwach und \Em[]klein.}
}
\footer{Die (für die Meisten) deutlich einfacher singbare da besser in den Stimmlagen von Mann und Frau liegende Version von Rosenstolz (2004) entsteht, indem Capo IV gesetzt wird. Hiervon existiert auch eine gezupfte Version für Gitarre, bei welcher \dq Half-Step-Down\dq ~gestimmt wird. Die Akkorde ergeben sich dann zu Am Dm G C F G Am in der Strophe und F C Dm Am F G Am im Refrain}
}
\end{document}

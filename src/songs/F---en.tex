\documentclass[../Liederbuch/LiederbuchGitarristen.tex]{subfiles}
\begin{document}
\song{F***en}{Das Niveau}{}{Deutsch}{Liedermacher}{

\verse{
\li{\E[]Viele uns'rer Lieder beginn'n mit \A[]dem Genuss der Biere.}
\li{Hier \D[]lieg ich nun - hab aufgehört zu \A[]zähl'n - eins, zwei, drei, viere.}
\li{\E[]Damit ist jetzt Schluss, denn \A[]es ist schon so spät,}
\li{dass sich \D[]alles nur noch um das \A[]Eine dreht.}
\li{Zu \E[]wem geh'n wir heut' nacht nach Hause, \A[]Männlein oder Weib?}
\li{Sind wir \D[]oben oder unten? \A[]Egal, wir sind bereit.}
\li{Für jedes \E[]Spielchen, jede Stellung, die \A[]Königin der Triebe.}
\li{Das \D[]Gefühl der Gefühle, \A[]klar, es geht um Liebe.}
}

\chorus{
\li{\E[]Ja, es ist so weit, die \G[]Stunde hat geschlagen,}
\li{das Ni\D[]veau darf endlich wieder \dq\A[]Ficken\dq\ sagen.}
\li{\E[]Ficken (Ficken!) was \G[]für ein schönes Wort:}
\li{\D[]Alles zwischen Liebe\A[]machen und Leistungssport.}
\li{Diese \E[]Lied ist nicht für Hörer unter \G[]18 geeignet.}
\li{\D[]Wehe, wer jünger ist und sein \A[]wahres Alter leugnet.}
\li{denn die \E[]Themen werden dreckig, die \G[]Sprache ordinär,}
\li{es geht\D[]… um \A[]Geschlechtsverkehr.}
}

\verse{
\li{Es \E[]gibt da eine Regel, die Gutes \A[]will und Böses schafft:}
\li{\D[]Nicht das verbotene F-Wort \A[]solang die Sonne lacht.}
\li{Wir \E[]soll'n sie nicht verderben, \A[]eure lieben Kleinen.}
\li{Haltet \D[]ihnen doch die Ohren zu und \A[]hört auf zu weinen.}
\li{\E[]Sex ist lebenswichtig, das ist \A[]unbenomm'n.}
\li{Kann mir mal \D[]bitte jemand sagen, wo die \A[]Kinder herkomm'n?}
\li{Dieses \E[]Lied hier steht für Freiheit und mehr \A[]Liebe auf der Welt.}
\li{\dq\D[]Gestatten, Das Niveau, hat hier jemand \A[]Sex bestellt?\dq}
}

\refrain

\verse{
\li{\E[]Vater im Himmel, \A[]vergib uns uns're Schuld,}
\li{wir \D[]üben uns doch jeden tag auf's \A[]Neue in Geduld.}
\li{Bis \E[]endlich der Schleier der \A[]Nacht sich auf uns legt}
\li{und \D[]sich in uns'rer Lendengegend \A[]wieder etwas regt.}
\li{Dann \E[]nur ein Blick, ein Nicken, die Nippel steh'n, die \A[]Nackenhaare auch}
\li{und \D[]fünf Minuten später komm'n wir \A[]grinsend aus 'nem Strauch.}
\li{\E[]Lustwandeln mit der Liebsten, \A[]wieder ohne Not,}
\li{wegen \D[]uns lockert der Papst das \A[]Kondomverbot.}
}

\refrain

\verse{
\li{\E[]Etwas in eig'ner Sache \A[]müssen wir noch sagen,}
\li{wem wir die \D[]Freiheit \dq Ficken\dq\ zu sagen \A[]zu verdanken haben.}
\li{Vor \E[]gut nem Vierteljahrtausend \A[]gab es ein paar Männer,}
\li{die benutzten ihren Verstand \dq Sapere \A[]aude, du Penner!\dq}
\li{Sie \E[]sagten, die Kirche habe \A[]nicht das Monopol}
\li{auf \D[]Seligkeit, auf Sex, auf Spaß \A[]und auf Alkohol.}
\li{Wir \E[]steh'n ohne Frage in \A[]deren Tradition,}
\li{durch \D[]uns kommt ihr zur Weisheit, nicht \A[]durch die Religion.}
}

\chorus{
\li{\E[]Ja, es ist so weit, das Licht der \G[]Wahrheit scheint heller,}
\li{\D[]niemand braucht zum \dq Ficken\dq -\A[]sagen in den Keller.}
\li{\E[]Ficken, Ficken!, \G[]schreit es raus.}
\li{Eure \D[]Seelen wollen atmen, \A[]also zeiht euch aus.}
\li{\E[]Habt ihr auf Männer oder Frauen \G[]oder beides Durst?}
\li{Ob ihr \D[]unter 18 seid beim Zuhör'n \A[]ist uns herzlich wurst.}
\li{Vielen \E[]Dank, Immanuel Kant, und \G[]danke, Voltaire,}
\li{\D[]euretwegen gibt es nicht die Hölle \A[]für Geschlechtsverkehr.}
}

\refrain

}
\end{document}
\documentclass[../Liederbuch/LiederbuchGitarristen.tex]{subfiles} 
\begin{document}
\song{Roter Mond}{Hortenring Ernsthofen}{}{Deutsch}{Bündisch/Pfadfinderlied}{
\verse{
\li{\Em[]Roter Mond \D[]über'm Silbersee, \Em[]Feuerglut \D[]wärmt den kalten Tee.}
\li{\brep\G[]Kiefernwald \D[]in der Nacht, und \Am[]noch ist der neue \Em[]Tag nicht erwacht.\erep}
}

\verse{
\li{\Em[]Sterne steh'n \D[]hell am Firmament, \Em[]solche Nacht \D[]findet nie ein End,}
\li{\brep\G[]Dieses Land, \D[]wild und schön, und \Am[]wir dürfen seine \Em[]Herrlichkeit seh'n.\erep}
}

\verse{
\li{\Em[]Rauer Fels, \D[]Moos und Heidekraut, \Em[]weit entfernt \D[]schon der Morgen graut,}
\li{\brep\G[]Fahne weht \D[]Gold auf Blau, das \Am[]Gras schimmert unterm \Em[]Morgentau.\erep}
}

\verse{
\li{\Em[]Fahrt vorbei, \D[]morgen geht es fort, \Em[]kommen wir \D[]wieder an den Ort,}
\li{\G[]Norden ist \D[]unser Glück, und \Am[]in uns bleibt nur Er\Em[]inn'rung zurück.}
\li{\G[]Norden ist \D[]unser Glück, und \Am[]wir schwören uns ein \Em[]neues Zurück.}
}
\footer{Das Lied entstand 1980 auf einer Schwedenfahrt. Im Orginal heißt es 'Fahne weht weiß auf grau' was die Wimpelfarben des Hortenrings Ernsthofen waren.}
}
\end{document}
\documentclass[../Liederbuch/LiederbuchGitarristen.tex]{subfiles} 
\begin{document}
\song{Heute hier, morgen dort}{T:Hannes Wader M:Gary Bolstad}{}{Deutsch}{Liedermacher}{
\verse{
\li{Heute \C[]hier, morgen dort, bin kaum \F[]da, muss ich \C[]fort}
\li{hab mich niemals des\Am[]wegen be\G[]klagt,}
\li{hab' es \C[]selbst so gewählt, nie die \F[]Jahre ge\C[]zählt,}
\li{nie nach \Am[]gestern und \G[]morgen ge\C[]fragt.}
}

\chorus{
\li{Manchmal \G[]träume ich schwer, und dann \F[]denk' ich, es \C[]wär'}
\li{Zeit zu \G[]bleiben und nun was ganz \F[]andres zu \C[]tun.}
\li{So vergeht Jahr um Jahr und es \F[]ist mir längst \C[]klar,}
\li{dass nichts bleibt, dass nichts \G[]bleibt, wie es \C[]war.}
}

\verse{
\li{Dass man \C[]mich kaum vermisst, schon nach \F[]Tagen ver\C[]gisst,}
\li{wenn ich längst wieder \Am[]anderswo \G[]bin,}
\li{stört und \C[]kümmert mich nicht, vielleicht \F[]bleibt mein Ge\C[]sicht}
\li{doch dem \Am[]ein' oder \G[]anderen im \C[]Sinn.}
}

\refrain

\verse{
\li{Fragt mich \C[]einer, warum ich so \F[]bin, bleib ich \C[]stumm,}
\li{denn die Antwort da\Am[]rauf fällt mir \G[]schwer,}
\li{denn was \C[]neu ist, wird alt, und was \F[]gestern noch \C[]galt,}
\li{stimmt schon \Am[]heut' oder \G[]morgen nicht \C[]mehr.}
}

\repref{2}
\footer{Die Melodie entstammt dem Song 'Indian Summer' des US-amerikanischen Musikers Gary Bolstad, der in den 1960er Jahren in Berlin studierte und in Folkclubs auftrat. Der deutsche Text stammt von Hannes Wader. Titel und Text des Liedes knüpfen an die Tradition und Lebenshaltung der Wandervogel-Bewegung des frühen 20. Jahrhunderts an.}
}
\end{document}
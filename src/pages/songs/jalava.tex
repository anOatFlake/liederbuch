\documentclass[../Liederbuch/LiederbuchGitarristen.tex]{subfiles} 
\begin{document}
\song{Jalava}{T:Heinz G. Unger M:Georg Herrnstadt/Wilhelm Resetarits}{}{Deutsch}{}{
\verse{               
\li{Von \Am[]Sonn' und Kessel schwarzgebrannt und auch vom scharfen \E[]Wind}
\li{steht Jalava am Führerstand, wo Dampf und Flammen \Am[]sind.}
\li{Sein \Dm[]neuer Heizer ist dabei, der \Am[]ihm die Flamme nährt,}
\li{auf der \E[]Lokomotive zwei-neun-drei, die \Am[]heut' nach Russland \E[7]fährt.}
\li{Ein \Am[]kleiner Mann von schmalem Bau, der werkt dort auf der \E[]Brücke,}
\li{Ruß im Gesicht, das Haar war grau, es war eine Pe\Am[]rücke.}
}
 
\chorus{
\li{\Dm[]Jalava, Jalava, du Finne, was \Am[]lachst du so gegen den Wind?}
\li{Ich \E[]lache, weil meine Sinne \Am[]alle beisammen sind.}
\li{Und \Dm[]weil wir weiterkamen, und \Am[]weil die Welt sich dreht,}    
\li{und \E[]weil mein Heizer von Flammen und \Am[]Dampfkesseln \E[]was ver\Am[]steht.}
\li{\Dm[]Jamba - Jambadada, \Am[]Jamba - Jambadada, \E[]Jamba - Jambadada, \Am[]Jamba-\A[]a-\A[7]ah!}
\li{\Dm[]Jamba - Jambadada, \Am[]Jamba - Jambadada, \E[]Jamba - Jambadada, \Am[]hej \E[]hej \Am[]hej!}
}

\verse{ 
\li{Sie \Am[]dampfen ein in Beloostrow, wo Schocks von Offi\E[]zieren}
\li{die Züge auf dem Grenzbahnhof penibel kontrol\Am[]lieren.}
\li{Sie \Dm[]prüfen jegliches Gesicht bei \Am[]ihrer Inspizierung,}
\li{doch \E[]sehen sie am Kessel nicht den \Am[]Staatsfeind der Re\E[7]gierung.}
\li{\Am[]Jalava weiß, worum es geht und langsam dampft vor\E[]bei}
\li{am letzten Posten, der dort steht, Lokomotive zwei-neun-\Am[]drei.}
}

\refrain

\verse{
\li{Jetzt \Am[]saust die Grenzstation vorbei, die Birken stehen \E[]nackt,}
\li{die Lokomotive zwei-neun-drei schnauft in erhöhtem \Am[]Takt.}
\li{Und \Dm[]Jalava lacht in den Wind, in \Am[]den Oktoberregen.}
\li{\E[]Heizer, wenn wir drüben sind, dann \Am[]wird sich was be\E[7]wegen!}
\li{Jetzt \Am[]schneidet der Oktoberwind die letzten Äpfel \E[]an,}
\li{die an den kahlen Bäumen sind an der finnischen Eisen\Am[]bahn.}
}

\chorus{
\li{\Dm[]Jalava, Jalava du Finne, was \Am[]lachst du so gegen den Wind?}
\li{Ich \E[]lache, weil meine Sinne \Am[]alle beisammen sind.}
\li{und \Dm[]weil uns die Fahrt in den Bahnhof \Am[]hinter der Grenze führt}
\li{und \E[]Wladimir Iljitsch Uljanow, mein \Am[]Heizer, die \E[]Flammen \Am[]schürt.}
\li{\Dm[]Jamba - Jambadada, \Am[]Jamba - Jambadada, \E[]Jamba - Jambadada, \Am[]Jamba-\A[]a-\A[7]ah!}
\li{\Dm[]Jamba - Jambadada, \Am[]Jamba - Jambadada, \E[]Jamba - Jambadada, \Am[]hej \E[]hej \Am[]hej!}
}
 
\footer{
Mit deutscher Unterstützung kehrte Wladimir Iljitsch Uljanow, genannt Lenin, im April 1917 per Eisenbahn aus dem Schweizer Exil über Deutschland, Schweden und Finnland nach Russland zurück. Die deutsche Oberste Heeresleitung hoffte, Lenin würde das krisengeschwächte Russland revolutionieren und dadurch Friedensverhandlungen ermöglichen. Im November kam er im Laufe der so genannten Oktoberrevolution an die Macht und legte den Grundstein für die spätere Sowjetunion. Zuvor musste er sich von August bist Oktober in Finnland verstecken. Der finnische Lokomotivführer Huge Jalava schmuggelte Lenin, als Heizer verkleidet, beide Male über die Grenze.
}
}
\end{document}
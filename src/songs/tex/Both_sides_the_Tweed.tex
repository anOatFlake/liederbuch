\documentclass[../Liederbuch/LiederbuchGitarristen.tex]{subfiles} 
\begin{document}
\song{Both sides the Tweed}{James Hogg, Dick Gaughan}{}{Englisch}{Irisch/Schottisch}{
\verse{
\li{What's the \Am[]spring breathing \C[]jasmine and \F[]rose, what's the \C[]summer with \Am[]all its gay \G[]train,}
\li{or the \Am[]splendour of \C[]autumn to \F[]those who've \C[]bartered their \G[]freedom for \Am[]gain?}
}

\chorus{
\li{Let the \F[]love of our land's sacred \Em[]rights to the \Am[]love of our \F[]people suc\G[]ceed\Em[].}
\li{Let \Am[]friendship and \C[]honour u\F[]nite and \C[]flourish on \G[]both sides the \Am[]Tweed.}
}

\verse{
\li{No \Am[]sweetness the \C[]senses can \F[]cheer which cor\C[]ruption and \Am[]bribery \G[]bind.}
\li{No \Am[]brightness that \C[]gloom can not \F[]clear, for \C[]honour's the \G[]sum of the \Am[]mind.}
}

\refrain

\verse{
\li{Let \Am[]virtue dis\C[]tinguish the \F[]brave, place \C[]riches in \Am[]lowest de\G[]gree.}
\li{Think them \Am[]poorest who \C[]can be a \F[]slave them \C[]richest who \G[]dare to be \Am[]free.}
}

\refrain
\footer{Das Lied behandelt die 1707 geschlossenen Acts of Union, in denen England und Schottland zu Großbritannien zusammengeschlossen wurden. Neben der versöhnlichen Grundstimmung wird dabei auch Korruption angesprochen, die in den Verhandlungen herrschte. Der Tweed ist ein Fluss der teilweise die Grenze zwischen Schottland und England bildet.}
}
\end{document}
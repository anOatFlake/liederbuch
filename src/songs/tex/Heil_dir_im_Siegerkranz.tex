\documentclass[../Liederbuch/LiederbuchGitarristen.tex]{subfiles}
\begin{document}
\song{Heil dir im Siegerkranz}{Heinrich Harries}{}{Deutsch}{}{

\verse{
\li{\F[]Heil \Dm[]dir \Gm[]im \C[7] Sieger\C[]kranz, \F[]Herr\Dm[]scher \Bb[]des \F[]Vater\Dm[]lands! \Gm[]Heil, \F[]Kai\C[]ser, \F[]dir!}
\li{\brep \F[]Fühl in des Thrones Glanz \Gm[]die hohe \Bb[]Wonne \C[]ganz,}
\li{\F[]Lieb\Gm[]ling des \F[]Volks zu \C[]sein, \Bb[]heil \F[]Kai\C[]ser \F[]dir! \erep}
}

\verse{
\li{\F[]Nicht \Dm[]Ross, \Gm[]nicht \C[7] Reisi\C[]ge \F[]si\Dm[]chern \Bb[]die \F[]steile \Dm[]Höh, \Gm[]wo \F[]Für\C[]sten \F[]stehn:}
\li{\brep \F[]Liebe des Vaterlands, \Gm[]Liebe des \Bb[]freien \Gm[]Manns}
\li{\F[]grün\Gm[]den den \F[]Herrscher\C[]thron \Bb[]wie \F[]Fels \C[]im \F[]Meer. \erep}
}

\verse{
\li{\F[]He\Dm[]ili\Gm[]ge \C[7]Flamme, \C[]glüh, \F[]glüh \Dm[]und \Bb[]er\F[]lösche \Dm[]nie \Gm[]fürs \F[]Va\C[]ter\F[]land!}
\li{\brep \F[]Wir alle stehen dann \Gm[]mutig für \Bb[]einen \C[]Mann,}
\li{\F[]käm\Gm[]pfen und \F[]bluten \C[]gern \Bb[]für \F[]Thron \C[]und \F[]Reich! \erep}
}

\verse{
\li{\F[]Han\Dm[]del \Gm[]und \C[7]Wissen\C[]schaft \F[]he\Dm[]ben \Bb[]mit \F[]Mut und \Dm[]Kraft \Gm[]ihr \F[]Haupt \C[]em\F[]por!}
\li{\brep \F[]Krieger- und Heldentat \Gm[]finden ihr \Bb[]Lorbeer\C[]blatt}
\li{\F[]treu \Gm[]aufge\F[]hoben \C[]dort \Bb[]an \F[]dei\C[]nem \F[]Thron! \erep}
}

\verse{
\li{\F[]Sei, \Dm[]Kai\Gm[]ser \C[7]Wilhelm, \C[]hier \F[]lang \Dm[]dei\Bb[]nes \F[]Volkes \Dm[]Zier, \Gm[]der \F[]Mensch\C[]heit \F[]Stolz!}
\li{\brep \F[]Fühl in des Thrones Glanz \Gm[]die hohe \Bb[]Wonne \C[]ganz,}
\li{\F[]Lieb\Gm[]ling des \F[]Volks zu \C[]sein, \Bb[]heil \F[]Kai\C[]ser \F[]dir! \erep}
}
\footer{Als eigentlicher Verfasser der deutschen Kaiserhymne \dq Heil dir im Siegerkranz\dq ~ist der schleswigsche Pfarrer Heinrich Harries (1762-1802) zu betrachten. Am 27. Januar 1790 veöffentlichte er im Flensburger Wochenblatt \dq Ein Lied für den dänischen Untertan, an seines Königs Geburtstag zu singen\dq ~in der Melodie des englischen Volksliedes \dq God save great George the King\dq, das mit den Worten beginnt: \dq Heil dir, dem liebenden Herrscher des Vaterlands! Heil, Christian dir!\dq ~Das Lied wurde dann von Balthasar Gerhard Schumacher auf fünf Strophen verkürzt und erschien entsprechend umgearbeitet in der Speyerschen Zeitung vom 17. Dezember 1793 als \dq Berliner Volksgesang\dq , der bald zur Nationalhymne werden sollte. \dq Heil unserem Fürsten (König), Heil\dq ~oder \dq Den König segne Gott\dq ~wurden dann auch in den anderen deutschen Staaten amtlich anerkannte Nationalhymnen.}
}
\end{document}
\documentclass[../Liederbuch/LiederbuchGitarristen.tex]{subfiles} 
\begin{document}
\song{Die Weber}{T:Heinrich Heine M:Jörg Ermisch}{}{Deutsch}{}{
\verse{
\li{\Em[]Im düstren \D[]Auge keine \Em[]Träne\sch{(}\C* \D* \D[sus4]* \D* \Em[)]}
\li{Sie sitzen am \D[]Webstuhl und fletschen die \Em[]Zähne.\sch{(}\C* \D* \D[sus4]* \D* \Em[)]}
\li{Deutschland, wir \D[]weben dein Leichen\G[]tuch!}
\li{\Em[]Wir weben hi\D[]nein in den 3-fachen \G[]Fluch!}
\li{Wir \Bm[]weben, wir \G[]we - \D[]e - \G[]ben!\sch{(}\D* \Em* \C* \D* \Em[)]}
}

\verse{
\li{\Em[]Ein Fluch dem \D[]Gotte zu dem wir \Em[]gebeten\sch{(}\C* \D* \D[sus4]* \D* \Em[)]}
\li{in Winters\D[]kälte und Hungers\Em[]nöten.\sch{(}\C* \D* \D[sus4]* \D* \Em[)]}
\li{Wir haben ver\D[]gebens gehofft und ge\G[]harrt, \Em[]man hat uns ge\D[]äfft, gefoppt und ge\G[]narrt!}
\li{Wir \Bm[]weben, wir \G[]we - \D[]e - \G[]ben!\sch{(}\D* \Em* \C* \D* \Em[)]}
}

\verse{
\li{\Em[]Ein Fluch dem \D[]König, dem König der \Em[]Reichen,\sch{(}\C* \D* \D[sus4]* \D* \Em[)]}
\li{den unser \D[]Elend nicht konnte er\Em[]weichen.\sch{(}\C* \D* \D[sus4]* \D* \Em[)]}
\li{Der den letzten \D[]Groschen von uns er\G[]presst \Em[]und uns wie \D[]Hunde erschießen \G[]lässt.}
\li{Wir \Bm[]weben, wir \G[]we - \D[]e - \G[]ben!\sch{(}\D* \Em* \C* \D* \Em[)]}
}

\verse{
\li{\Em[]Ein Fluch dem \D[]falschen Vater\Em[]lande,\sch{(}\C* \D* \D[sus4]* \D* \Em[)]}
\li{wo nur ge\D[]deihen Schmach und \Em[]Schande,\sch{(}\C* \D* \D[sus4]* \D* \Em[)]}
\li{wo jede \D[]Blume früh ge\G[]knickt, \Em[]wo Fäulnis und \D[]Moder den Wurm er\G[]quickt.}
\li{Wir \Bm[]weben, wir \G[]we - \D[]e - \G[]ben!\sch{(}\D* \Em* \C* \D* \Em[)]}
}

\verse{
\li{\Em[]Das Schiffchen \D[]fliegt, der Webstuhl \Em[]kracht,\sch{(}\C* \D* \D[sus4]* \D* \Em[)]}
\li{wir weben \D[]emsig Tag und \Em[]Nacht\sch{(}\C* \D* \D[sus4]* \D* \Em[)]}
\li{Altdeutschland wir \D[]weben dein Leichen\G[]tuch. \Em[]Wir weben hi\D[]nein den 3-fachen \G[]Fluch.}
\li{Wir \Bm[]weben, wir \G[]we - \D[]e - \G[]ben! Wir \Bm[]weben, wir \G[]we - \D[]e - \G[]ben!\sch{(}\D* \Em* \C* \D* \Em[)]}
}
\footer{Das Lied basiert auf dem Gedicht 'Die schlesischen Weber' von Heinrich Heine. Es handelt vom Elend der schlesischen Weber, die 1844 einen Aufstand gegen Ausbeutung und Lohnverfall wagten und damit auf die im Rahmen der Industrialisierung entstandenen Missstände aufmerksam machten. Das Königlich Preußische Kammergericht verbot das Gedicht wegen seines 'aufrührerischen Tones'.}
}
\end{document}
\documentclass[../Liederbuch/LiederbuchGitarristen.tex]{subfiles} 
\begin{document}
\song{In dem dunklen Wald von Paganowo}{M:Matwei Isaakowitsch Blanter}{}{Deutsch}{}{
\verse{
\li{\Em[]In dem dunklen Wald von Paga\B[7]nowo}
\li{lebte einst ein wilder Räubers\Em[]mann.}
\li{\brep\ch{(} \Em[]Und \C[)]er \G[]war der \Am[]Schrecken aller \Em[]Guten,}
\li{\Am[]weil er viel \Em[]Böses \B[7]hatte schon ge\Em[]tan.\erep}
}

\verse{
\li{\Em[]Doch da kam der lange Leutnant \B[7]Nagel,}
\li{und er sprach: \dq Ich fass' ihn mir beim \Em[]Bart!\dq}
\li{\brep\ch{(}\g{ }\Em[]Und \C[)]er \G[]hatt' eine \Am[]kühne Schar von \Em[]Häschern}
\li{\Am[]um sich he\Em[]rum ge\B[7]schart zu kühner \Em[]Tat.\erep}
}

\verse{
\li{\Em[]In den dunklen Wald von Paga\B[7]nowo}
\li{brach er ein bei Tag und auch bei \Em[]Nacht,}
\li{\brep\ch{(}\g{ }\Em[]bis \C[)]er \G[]dann den \Am[]frechen Räuber\Em[]burschen}
\li{\Am[]eines \Em[]Tags zur \B[7]Strecke hat ge\Em[]bracht.\erep}
}

\verse{
\li{\Em[]Und der Räuber, ja, der trug ein \B[7]Holzbein,}
\li{war ein richt'ger Mörder ja so\Em[]gar.}
\li{\brep\ch{(}\g{ }\Em[]Und \C[)]er \G[]musst' sich \Am[]selbst die Grube \Em[]graben,}
\li{\Am[]was seine \Em[]letzte \B[7]Räuberhandlung \Em[]war.\erep}
}

\verse{
\li{\Em[]Tot liegt nun im Wald von Paga\B[7]nowo}
\li{der verfluchte, wüste Räuber\Em[]hund.}
\li{\brep\ch{(}\g{ }\Em[]Und \C[)]das \G[]Lied vom \Am[]langen Leutnant \Em[]Nagel}
\li{\Am[]geht nun in \Em[]Russland \B[7]um von Mund zu \Em[]Mund.\erep}
}
\footer{Die Melodie stammt von Matwei Blanter, der sie 1938 zu dem Liebeslied 'Katjuscha' verfasste.}
}
\end{document}
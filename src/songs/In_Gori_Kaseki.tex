\documentclass[../Liederbuch/LiederbuchGitarristen.tex]{subfiles} 
\begin{document}
\song{In Gori Kaseki}{Alo (Alfons Hamm)}{}{Deutsch}{}{
\verse{
\li{In \Am[]Gori Kaseki, am \C[]Rande der Straße,}
\li{liegt \E[]einsam ein Knabe, der\ch{(} \Am[]regt sich \E[)]nimmer\Am[]mehr.}
\li{Gefährliche Straße, un\C[]heimliche Weite,}
\li{da\E[]rüber die Wolken sind\ch{(} \Am[]wie ein \E[)]Geister\Am[]heer.}
}

\chorus{
\li{\Dm[]Zweitausend \Am[]Reiter \E[]fegen he\Am[]ran, ja heran.}
\li{\Dm[]Was sind da\Am[]gegen einhundert\E[]fünfundachtzig\ch{(} \Am[]Mann?\E[)]* \Am}
}

\verse{
\li{In \Am[]Gori Kaseki, sind \C[]nur noch Ruinen}
\li{von \E[]Hütten, die bleiben, kein\ch{(} \Am[]Laut, sie \E[)]stehen \Am[]leer.}
\li{Geflohen die Dörfler, die \C[]Hunde, die Katzen,}
\li{es \E[]blieben nur Ratten bei\ch{(} \Am[]dem ge\E[)]schmolzenen \Am[]Heer.}
}

\refrain

\verse{
\li{In \Am[]Gori Kaseki ver\C[]brannt ist die Erde,}
\li{ver\E[]giftet die Brunnen, hier\ch{(} \Am[]fängt die \E[)]Hölle \Am[]an.}
\li{Gefrorene Brote mit \C[]Beilen sie teilen}
\li{und \E[]Schneewasser nur dem\ch{(} \Am[]todge\E[)]weihten \Am[]Mann.}
}

\refrain

\verse{
\li{Erst \Am[]schicken sie Frauen, dann \C[]Kinder und Greise,}
\li{sie \E[]liegen im Eise, ein\ch{(} \Am[]schneebe\E[)]deckter \Am[]Wall.}
\li{Darüber sie sprengen, gleich \C[]wilden Gesängen,}
\li{mit \E[]Unrast die Reiter, des\ch{(} \Am[]To\g{-}\E[)]des Va\Am[]sall.}
}

\refrain

\verse{
\li{In \Am[]Gori Kaseki sind \C[]alle geblieben - }
\li{zwei\E[]tausendeinhundertun\ch{(}d\Am[]fünfund\E[)]achtzig \Am[]Mann.}
\li{Vom Himmel kam Feuer, da\C[]rin sind sie geblieben,}
\li{zwei\E[]tausendeinhundertun\ch{(}d\Am[]fünfund\E[)]achtzig \Am[]Mann.}
}

\refrain
\footer{Gory Kaseki liegt in den Waldaihöhen 200 km westlich von Moskau.
Aus einem Tondokument aus dem Jahre 1950:
\dq Das Lied von Gori Kaseki ist eigentlich, wenn man so sagen will, stellvertretend für Situationen dieser Art, wie sie in dem Lied beschrieben werden, wie man sie allgemein damals als Soldat im Ostfeldzug, also in Russland erlebte. Ich muss dazu sagen, dass es in Russland damals tatsächlich noch richtige Soldaten auf Pferden gegeben hat, allerdings in Gori Kaseki waren es keine 2000 Reiter und man soll deshalb gar nicht enttäuscht sein darüber. Es waren Panzerwagen gewesen, die uns niedermachten. Aber es ist im Russlandfeldzug noch die Kavallerie in Aktion getreten auf beiden Seiten und vor allen Dingen bei den Russen und weil es sich vom Lied her so besser machte, ist es in diese Form gebracht worden, so wie es vom Text her bekannt ist, dieses Gori Kaseki Lied...\dq}
}
\end{document}
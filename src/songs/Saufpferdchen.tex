\documentclass[../Liederbuch/LiederbuchGitarristen.tex]{subfiles}
\begin{document}
\song{Saufpferdchen}{Blutjungs}{}{Deutsch}{Splatterpop}{

\intro{
\chli{\E[5] \C \D}
}

\verse{
\li{\E[5]Aha, du bist immer noch - auf der Flucht vor'm Promille-Loch}
\li{Schleppst dich von Bar zu Bar doch eins ist völlig klar:}
}

\chorus{
\li{\NC Du kannst nicht saufen\E[5]! Nein nein, das kannst du nicht}
\li{Es ist nicht \G[]cool wer sich - auf die \A[]Klamotten bricht}
\li{\NC Du kannst nicht saufen\E[5]!}
\li{Wer sich auf's Top oder Kleid verdaute \C[]Brocken \D[]speit tut mir nicht \E[5]leid}
}

\verse{
\li{\E[5]Dein Hang zu Kümm und Korn weckt in dir Hass und Zorn}
\li{Du neigst zu Pöbelei'n denn mehr fällt dir nicht ein}
}

\chorus{
\li{\NC Du kannst nicht saufen\E[5]! Nein nein, das kannst du nicht}
\li{Es ist nicht \G[]stark wer angesoffen and'ren \A[]Jochbeine bricht}
\li{\NC Du kannst nicht saufen\E[5]!}
\li{Du aggressiver Vollidiot! Ich hoffe \C[]sehr es schlägt dich \D[]dereinst einer \E[5]tot.}
}

\bridge{
\chli{\E[5]}
}

\verse{
\li{\E[5]In deiner trunk'nen Welt bist du ein Frauenheld}
\li{Sei tapfer armer Tor, es kommt dir nur so vor}
}

\chorus{
\li{\NC Du kannst nicht saufen\E[5]! Nein nein, das kannst du nicht}
\li{Wer Frau'n \G[]betrunken antatscht ist und bleibt ein \A[]ganz armer Wicht}
\li{\NC Du kannst nicht saufen\E[5]!}
\li{Denkst du auch mit dem Lurch. Ich wünsch' dir \G[]jede zieht die \A[]Rückhand richtig \E[5]durch}
\li{\A[]Anstand und \B[]Moral liegen \E[5]brach. \A[]Mach' erstmal dein Saufpferdchen \B[]nach.}
}

\chorus{
\li{\NC Du kannst nicht saufen\E[5]! Nein nein, das kannst du nicht}
\li{Was muss pas\G[]sieren, dass ein Depp wie du sein \A[]Saufpferdchen kriegt?}
\li{\NC Du kannst nicht saufen\E[5], drum' lass es besser sein}
\li{und trete \C[]ein in einen \D[]Temperenzverein\E[5]}
}

\footer{
Ein Temperenzverein (lat. temperantia \dq Mäßigung\dq ) setzt sich gegen den Konsum von Alkohol ein. Die erste Abstinenzbewegung fand ihren Anfang 1829 in Irland. Durch solche Vereine konnten sich zu den Hochzeiten der Bewegung - Ende 19. Anfang 20. Jhd. - auch gerade Frauen trotz fehlender politischer Rechte Gehör verschaffen. In mehreren Ländern wurden Gesetze zur Prohibition von Alkohol erlassen.  
}

}
\end{document}
\documentclass[../Liederbuch/LiederbuchGitarristen.tex]{subfiles} 
\begin{document}
\song{Er war ein Pfadfinder}{trad.}{}{Deutsch}{Bündisch/Pfadfinderlied}{
\chorus{
\li{Er war ein \Em[]Pfadfinder, von kernigem Schliff, er hielt stets die Treue,}
\li{was keiner begriff. So viele Vereine, die \Am[]lockten ihn raus,}
\li{Doch die Pfadfinderkluft, die \Em[]zog er nicht aus.}
}

\verse{
\li{Mit \Em[]zwölf Jahren fing er als Jungpfadfinder an, da war er schon bekannt bei Kornett}
\li{und Kaplan. Er kannte die Gesetze von \Am[]Baden Powell, er kannte sie \B[7]alle very \Em[]well. - \refrain*}
}

\verse{
\li{Zu\Em[]hause schlief er stets unter dem Bett. Die Folge davon, er wurde Kornett.}
\li{Die Sippe kaufte ihm \Am[]Schaumgummi ein, doch er schlief \B[7]lieber auf Schotterge\Em[]stein. - \refrain*}
}

\verse{
\li{Und \Em[]selbstverständlich war er auch Sippensuppenkoch, versalzte jede Suppe noch und noch,}
\li{vor zwei Pfund Salz machte \Am[]er nicht halt, selbst Regen\B[7]würmer ließen ihn \Em[]kalt. - \refrain*}
}

\verse{
\li{Der \Em[]guten Taten tat er stets zu viel. Brachte Damen über'n Fahrdamm}
\li{wie Weiland Harry Peel. Und brach sich eine Dame die \Am[]Knochen dabei,}
\li{so schiente er sie \B[7]gleich als gute Tat Nummer \Em[]zwei. - \refrain*}
}

\verse{
\li{Und \Em[]als er mit 17 Rover war, da schor er sich zum ersten Mal den blonden Flaumenbart.}
\li{Er zog lange Hosen an und \Am[]tanzt im Boogie-Schritt.}
\li{Das Fahrtenmesser \B[7]nahm er stets im Sockenhalter \Em[]mit. - \refrain*}
}

\verse{
\li{Und \Em[]als er mit 20 Feldmeister war, da liebte er ein Mädchen mit strohblondem Haar.}
\li{Er liebte sie sehr, doch sie \Am[]war ihm nicht treu,}
\li{da widmete er sich \B[7]wieder der Pfadfinde\Em[]rei. - \refrain*}
}

\verse{
\li{Am \Em[]30. Mai kratzte er sich am Bein. Mit Blutvergiftung ging er in die Jagdgründe ein.}
\li{Und Baden Powell stand am \Am[]Himmelstor. Zur Begrüßung \B[7]sang der englische \Em[]Chor:}
}

\chorus{
\li{Du warst ein \Em[]Pfadfinder von kernigem Schliff, du hieltst stets die Treue,}
\li{was keiner begriff. So viele Vereine, die \Am[]lockten dich raus,}
\li{doch die Pfadfinderkluft, die \Em[]zogst du nie aus.}
}
\footer{Harry Piel (1892-1963) war ein deutscher Regisseur und Film-Schauspieler. Mit dem Film 'Der große Unbekannte' (1919) begann er unter dem Namen 'Harry Peel' auch international bekannt zu werden.}
}
\end{document}
\documentclass[../Liederbuch/LiederbuchGitarristen.tex]{subfiles} 
\begin{document}
\song{Der Pfahl}{T:Oskar Kröher M:Lluís Llach}{}{Deutsch}{}{
\verse{
\li{\Em[]Sonnig be\B[7]gann es zu \Em[]Tagen, ich stand ganz früh vor der \B[7]Tür,}
\li{\Em[]sah nach den \B[7]fahrenden \Em[]Wagen, da sprach Alt-\B[7]Siset zu \Em[]mir:}
\li{\dq Siehst du den brüchigen \B[7]Pfahl dort, \Em[]mit unser'n Fesseln um\B[7]schnürt?}
\li{\Em[]Schaffen wir \B[7]doch diese \Em[]Qual fort, ran an ihn, \B[7]dass er sich \Em[]rührt.\dq}
}

\chorus{
\li{\Em[]Ich drücke \B[7]hier, und du ziehst \Em[]weg,}
\li{so kriegen \B[7]wir den Pfahl vom \Em[]Fleck,}
\li{werden ihn \Am[]fällen, fällen, \Em[]fällen,}
\li{werfen ihn \B[7]morsch und faul zum \Em[]Dreck.}
\li{Erst wenn die \B[7]Eintracht uns be\Em[]wegt,}
\li{haben wir \B[7]ihn bald umge\Em[]legt,}
\li{und er wird \Am[]fallen, fallen, \Em[]fallen,}
\li{wenn sich ein \B[7]jeder von uns \Em[]regt.}
}

\verse{
\li{\dq\Em[]Ach, Siset, \B[7]noch ist es \Em[]nicht geschafft, an meiner Hand platzt die \B[7]Haut.}
\li{\Em[]Langsam auch \B[7]schwindet schon \Em[]meine Kraft, er ist zu \B[7]mächtig ge\Em[]baut.}
\li{Wird es uns jemals ge\B[7]lingen? \Em[]Siset, es fällt mir so \B[7]schwer!\dq}
\li{\dq\Em[]Wenn wir das \B[7]Lied nochmal \Em[]singen, geht es viel \B[7]besser, komm \Em[]her!\dq}
}

\refrain

\verse{
\li{\Em[]Der alte \B[7]Siset sagt \Em[]nichts mehr, böser Wind hat ihn ver\B[7]weht.}
\li{\Em[]Keiner weiß \B[7]von seiner \Em[]Heimkehr, keiner weiß \B[7]wie es ihm \Em[]geht.}
\li{Alt-Siset sagte uns \B[7]allen, \Em[]hör es auch du, krieg es \B[7]mit:}
\li{\Em[]Der morsche \B[7]Pfahl wird schon \Em[]fallen, wie es ge\B[7]schieht in dem \Em[]Lied.}
}

\refrain
\footer{
\dq Der Pfahl\dq{} ist ein katalanisches Kampflied gegen die Franco-Diktatur. Es wurde 1968 von Lluís Llach geschrieben.
Der Pfahl (l'estaca) steht darin sinnbildlich für den Staat (l'estat), den es zu stürzen gilt.
}
}
\end{document}
\documentclass[../Liederbuch/LiederbuchGitarristen.tex]{subfiles} 
\begin{document}
\song{Wir lagen vor Madagaskar}{Just Scheu}{}{Deutsch}{}{
\verse{
\li{Wir \C[]lagen vor Madagaskar und \G[]hatten die \G[7]Pest an \C[]Bord,}
\li{in den Kesseln, da faulte das Wasser, und \G[]täglich ging \G[7]einer über \C[]Bord.}
}

\chorus{
\li{Ahoi, Kameraden, a\G[]hoi, a\C[]hoi!}
\li{Leb wohl, kleines Madel, leb \G[]wohl, leb \C[]wohl!}
\li{Ja, \C[7]wenn das \ch{F}Schifferklavier an \C[]Bord ertönt,}
\li{ja, dann sind die Matrosen so \G[]still, ja, so \G[7]still,}
\li{weil ein \C[]jeder nach seiner Heimat sich sehnt,}
\li{die er \G[]gerne einmal \G[7]wiedersehen \C[]will.}
}

\verse{
\li{Wir \C[]lagen schon 14 Tage, kein \G[]Wind durch die \G[7]Segel uns \C[]pfiff.}
\li{Der Durst war die größte Plage, da \G[]liefen wir \G[7]auf ein \C[]Riff.}
}

\refrain

\verse{
\li{Der \C[]lange Hein war der Erste, er \G[]soff von dem \G[7]faulen \C[]Nass.}
\li{Die Pest gab ihm das Letzte, und \G[]wir ihm ein \G[7]Seemanns\C[]grab.}
}

\refrain
\footer{Als Entstehungsjahr gilt 1934. Allgemein wird das Lied als Seemannslied angesehen, ist jedoch in den 1930er-Jahren auch als Fahrtenlied bekannt gewesen. Es erschien in den Liederbüchern von Gruppen der verbotenen bündischen Jugend (Edelweißpiraten), die wegen der Übernahme solchen Liedguts in das Visier der Gestapo kam. Man vermutet, dass das Lied die Geschehnisse aus der Zeit des Russisch-Japanischen Kriegs (1904/1905) beschreibt. Russische Schiffe mussten wegen Reparaturen unfreiwilligen Aufenthalt vor Madagaskar einlegen. Zeitraum und Begleitumstände (Krankheit, Tod) sind mit dem Text des Liedes deckungsgleich.}
}
\end{document}
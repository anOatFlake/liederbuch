\documentclass[../Liederbuch/LiederbuchGitarristen.tex]{subfiles} 
\begin{document}
\song{Scherenschleiferweise}{Fritz Grasshoff}{}{Deutsch}{Mittelalterlich}{
\verse{
\li{\F[]Sommers durch die \C[]Dörfer streifen, wenn \Gm[]die roten \Dm[]Beeren reifen}
\li{\F[]Und den Leuten \C[]Scheren schleifen, \Bb[]Messer, Scheren, \A[]Klingen}
\li{\F[]Sommers durch die \C[]Dörfer streifen, \Gm[]Mädchen in die \Dm[]Röcke greifen}
\li{\F[]Küssen, in den \C[]Pöter kneifen, \Bb[]lachen, lieben, \A[]singen.\A[7]}
}

\chorus{
\li{\F[]Und das Rädchen schnurren lassen, \C[]surren lassen, \Dm[]gurren lassen}
\li{\F[]Frech das Glück beim Schopfe fassen \C[]und den Kopf nicht \Dm[]hängen lassen}
\li{\F[]Und das Rädchen schnurren lassen, \C[]surren lassen, \Dm[]gurren lassen}
\li{\F[]Schenk voll ein und hoch die Tassen, \C[]nie den Magen \Dm[]knurren lassen}
\li{\Bb[]Auf der langen \Dm[]Tippel\A[]reise\A[7],{ } \Bb[]das ist Scheren\C[]schleifer\Dm[]weise}
\li{\Bb[]Auf der langen \Dm[]Tippel\A[]reise\A[7],{ } \Bb[]das ist Scheren\C[]schleifer\Dm[]weise}
}

\verse{
\li{\F[]Winters in Ta\C[]vernen hucken, \Gm[]viele kleine \Dm[]Schnäpse schlucken}
\li{\F[]Spät sich erst ins \C[]Bett verdrucken, \Bb[]lärmen und kr\A[]akeelen}
\li{\F[]Winters in Ta\C[]vernen hucken, \Gm[]rauchen, an den \Dm[]Ofen spucken}
\li{\F[]Andern in die \C[]Karten gucken, \Bb[]schnorren, betteln, \A[]stehlen\A[7]}
}

\repref{2}
}
\end{document}
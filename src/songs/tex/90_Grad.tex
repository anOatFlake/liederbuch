\documentclass[../Liederbuch/LiederbuchGitarristen.tex]{subfiles}
\begin{document}
\song{90 Grad}{Bodo Wartke}{}{Deutsch}{}{

\verse{
\li{Ich halte dich im \Bm[]Arm und spüre deine Haut.}
\li{Sie ist weich und \D[]warm und mir wundersam vertraut.}
\li{Noch nie hat Ver\Em[]gleichbares meine Hand umschmeichelt.}
\li{Wer Cashmere für \G[]weich hält, hat dich nie \A[]gestreichelt.}
}

\verse{
\li{Ich riech den \Bm[]Duft von deinem vollen Haar. Dein Haar, ja, es \D[]duftet so wunderbar.}
\li{Ich gerate \Em[]geradezu in Extase, dringt mir dein be\G[]törendes Odeur in die \A[]Nase.}
}

\verse{
\li{Ich spür bis zu den \Bm[]Ohren mein Herz leise klopfen}
\li{und aus meinen \D[]Poren heiße Schweißtropfen tropfen,}
\li{als sich unsere \Em[]Körper aneinander anzuschmiegen wagen}
\li{und in sanften \G[]Wogen sich bewegen und zu \A[]wiegen wagen.}
}

\verse{
\li{Wir sind \Bm[]vollkommen synchron. Ich glaub, das nennt man \D[]nonverbale Kommunikation.}
\li{Voller Grazie und \Em[]Anmut sind all deine Bewegungen,}
\li{wes\G[']wegen, zugegeben, ich ein wenig in Verlegenheit}
\li{und auch nicht unerheblich in \A[]Erregung bin.}
}

\chorus{
\li{Wir \G[6]sind im Rausch der Sinne, ich halt kurz inne, um dich \D[6]anzusehn. Du bist so wunderschön!}
\li{So \G[6]wie der Augenblick, den wir \E[7]gemeinsam teilen.}
\li{Ein Augen\A[]blick, von dem ich wünsch\Fs[7]te, er würde noch etwas ver\Fs[sus4]weilen...\Fs}
}

\verse{
\li{Ich glaub, so schön wie mit \Csm[]dir war es noch nie.}
\li{Komm, zelebrier mit \E[]mir unsere Harmonie!}
\li{Denn hierbei zeigt sich ge\Fsm[]nau, wir beide passen zusammen.}
\li{Du bist die \A[]Frau und ich bin der \B[]Mann.}
}

\verse{
\li{Ich fühle mich wie be\Csm[]freit von aller Last meines Seins.}
\li{Wir sind zu \E[]zweit und dabei gleichzeitig eins, entgegenge\Fsm[]setzt, doch beide Teil eines Ganzen,}
\li{wenn wir so wie \A[]jetzt\B[]... miteinander \E[]tanzen.}
}

\bridge{
\chli{\Csm \E \Fsm \A \B \rep{4}}
\chli{\Csm}
}

\verse{
\li{Ich fühle mich wie be\Csm[]freit von aller Last meines Seins.}
\li{Wir sind zu \E[]zweit und gleichzeitig eins,}
\li{entgegenge\Fsm[]setzt, doch beide Teil eines Ganzen,}
\li{wenn wir so wie \A[]jetzt\B[]...\Csm[]}
}

}
\end{document}
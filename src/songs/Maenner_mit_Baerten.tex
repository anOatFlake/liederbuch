\documentclass[../Liederbuch/LiederbuchGitarristen.tex]{subfiles} 
\begin{document}
\song{Männer mit Bärten}{T:Gottfried Wolters M:trad. flämisch}{}{Deutsch}{}{
\verse{
\li{\brep\Em[]Alle, die mit uns auf \Am[]Kaperfahrt \Em[]fahren}
\li{\Em[]Müssen Männer mit \Am[]Bärten \Em[]sein\erep}
}

\chorus{
\li{\G[]Jan und Hein und Klaas und Pit - \Em[]die haben \Am[]Bärte, \D[]die haben \Em[]Bärte}
\li{\G[]Jan und Hein und Klaas und Pit - \Em[]die haben \Am[]Bärte, die \D[]fahren \Am[]mit}
}

\verse{
\li{\brep\Em[]Alle, die Weiber und \Am[]Branntwein \Em[]lieben, \Em[]müssen Männer mit \Am[]Bärten \Em[]sein.\erep{ }- \refrain*}
}

\verse{
\li{\brep\Em[]Alle, die mit uns das \Am[]Walross \Em[]töten, \Em[]müssen Männer mit \Am[]Bärten \Em[]sein.\erep{ }- \refrain*}
}

\verse{
\li{\brep\Em[]Alle, die keinen Kla\Am[]bautermann \Em[]fürchten, \Em[]müssen Männer mit \Am[]Bärten \Em[]sein.\erep{ }- \refrain*}
}

\verse{
\li{\brep\Em[]Alle, die öligen \Am[]Zwieback \Em[]lieben, \Em[]müssen Männer mit \Am[]Bärten \Em[]sein.\erep{ }- \refrain*}
}

\verse{
\li{\brep\Em[]Alle, die Tod und \Am[]Teufel nicht \Em[]fürchten, \Em[]müssen Männer mit \Am[]Bärten \Em[]sein.\erep{ }- \refrain*}
}

\verse{
\li{\brep\Em[]Alle, die endlich zur \Am[]Hölle mit \Em[]fahren, \Em[]müssen Männer mit \Am[]Bärten \Em[]sein.\erep{ }- \refrain*}
}
\footer{Das ursprünglich flämische Lied \dq Al die willen te kapren varen\dq{} wurde in der ersten Hälfte des 19. Jahrhunderts in Dünkirchen aufgezeichnet. Die dortigen Einwohner verstanden sich als Freibeuter, die spanische, englische und niederländische Schiffe plünderten.}
}
\end{document}
\documentclass[../Liederbuch/LiederbuchGitarristen.tex]{subfiles} 
\begin{document}
\song{Tourdion}{Pierre Attaingnant}{}{Französisch}{Mittelalterlich}{
\verse{
\li{Sopran:}
\li{\brep\Dm[]Quand je bois du vin clairet, amis, tout \C[]tourne, tourne, tourne, \Dm[]tourne,}
\li{\Dm[]aussi désormais je bois Anjou \Am[]ou Ar\Dm[]bois.\erep}
\li{\brep\Dm[]Chantons et buvons, à \C[]çe flaçon faisons la \Dm[]guerre,}
\li{\Dm[]chantons et buvons, mes amis, \Am[]buvons \Dm[]donc.\erep}
}

\verse{
\li{Alt:}
\li{\brep\Dm[]Le bon vin nous \C[]a ren\Dm[]du gais,}
\li{\Dm[]chantons, oublions nos pei\Am[]nes, chan\Dm[]tons.\erep}
\li{\brep\Dm[]En mangeant d'un \C[]gras jam\Dm[]bon, à}
\li{\Dm[]çe flaçon fai\Am[]sons la \Dm[]guerre.\erep}
}

\verse{
\li{Tenor:}
\li{\Dm[]Buvons bien, là \C[]buvons \Dm[]donc, à}
\li{\Dm[]çe flaçon fai\Am[]sons la \Dm[]guerre.}
\li{\Dm[]Buvons bien, là \C[]buvons \Dm[]donc, a-}
\li{\Dm[]mis, trinquons, gaie\Am[]ment chan\Dm[]tons.}
\li{\brep\Dm[]En mangeant d'un \C[]gras jam\Dm[]bon, à}
\li{\Dm[]çe flaçon fai\Am[]sons la \Dm[]guerre.\erep}
}

\verse{
\li{Bass:}
\li{\Dm[]Buvons bien, bu\C[]vons mes a\Dm[]mis, trin-}
\li{\Dm[]quons, buvons, vi\Am[]dons nos \Dm[]verres.}
\li{\Dm[]Buvons bien, bu\C[]vons mes a\Dm[]mis, trin-}
\li{\Dm[]quons, buvons, gaie\Am[]ment chan\Dm[]tons.}
\li{\brep\Dm[]En mangeant d'un \C[]gras jam\Dm[]bon, à}
\li{\Dm[]çe flaçon fai\Am[]sons la \Dm[]guerre.\erep}
}

\footer{Tourdion ist ein französisches Trinklied aus dem 16. Jahrhundert. Der Text wurde nicht überliefert und stellt eine Rekonstruktion dar, weswegen nicht klar ist, ob es sich auch ursprünglich um ein Trinklied handelte. Die Melodiestimme ist der Sopran, die anderen Stimmen haben nur begleitende Funktion. Der zugehörige Kreispaartanz ist auf Mittelaltermärkten beliebt.}
}
\end{document}
\documentclass[../Liederbuch/LiederbuchGitarristen.tex]{subfiles}
\begin{document}
\song{Biblis Baby}{Carlos Mogutseu}{3}{Deutsch}{}{

\intro{
\chli{\brep\Am \Am \F \E\erep}
}

\verse{
\li{\Am[]Was du mit mir getan hast, machten \F[]andere Mädchen \E[]nie,}
\li{du \Am[]brachtest in mein Leben diese \F[]neue Ener\E[]gie.}
\li{Du \Am[]hast mich so verrückt gemacht und \F[]unvorstellbar \E[]scharf,}
\li{auch \F[]wenn ich nur in Strahlenschutzbe\E[]kleidung zu dir darf.}
}

\chorus{
\li{Biblis \Am[]Baby, \C[]komm zurück, ich \G[]brauche dich.}
\li{Biblis \Am[]Baby, \C[]bitte bitte \G[]strahl' für mich.}
\li{\C[]Dass du mich ver\G[]lassen hast, hab' \Am[]ich dir nie ver\E[]ziehen.}
\li{Wer \F[]lädt mir meine \E[]Autobatte\Am[]rie?}
}

\bridge{
\chli{\brep\Am \Am \F \E\erep\ \Am}
}

\verse{
\li{Dein \Am[]Haus stand am Reaktor und der \F[]Blick war wunder–\E[]schön,}
\li{vom \Am[]Küchenfenster konnten wir die \F[]Kühlkreisläufe \E[]seh'n.}
\li{Was \Am[]heiß und wild begonnen hat, ist \F[]lange schon \E[]vorbei.}
\li{Ein \F[]and'rer Kerl spielt sicher schon an \E[]deinem Slip aus Blei.}
}

\refrain

\bridge{
\chli{\brep\Am \Am \F \E\erep}
}

\verse{
\li{\Am[]Unser Urlaub war so herlich, vierzehn Tage Mühlheim-Kärlich.}
\li{\F[]Wir stachen Reifen platt an \E[]Biodieselwagen,}
\li{\Am[]wir lebten wilde Träume, spritzen Obst und fällten Bäume.}
\li{\F[]Wir warfen Ziegelsteine \E[]auf Solaranlagen}
\li{und wir \C[]badeten im \G[]Rhein und sangen \Am[]– obszöne Lieder.}
\li{Wir \C[]brannten in der Innenstadt zwei \G[]Greenpeace-Stände nieder.}
}

\refrain

\footer{
Der kleine Ort Biblis mit seinen 9073 Einwohnern (Stand: 2018) ist vor allem durch sein Kernkraftwerk bekannt, welches dort seit dem 16.07.1974 arbeitete. Bis zum 18.03.2011 konnte es bis zu 2525 Megawatt in das Stromnetz speisen. Es wurde dann im Zuge des sog. Atom-Moratoriums als Reaktion auf die Havarie des Kraftwerks in Fukushima stillgelegt und wird seit dem 01.06.2017 zurück gebaut. Der ehemalige hessische Ministerpräsident Roland Koch, zu diesem Zeitpunkt Vorstandschef des Baukonzerns Bilfinger, forderte 2013 dazu auf, Steine auf Photovoltaikanlagen zu werfen.
}

}
\end{document}
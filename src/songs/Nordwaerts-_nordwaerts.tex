\documentclass[../Liederbuch/LiederbuchGitarristen.tex]{subfiles} 
\begin{document}
\song{Nordwärts, nordwärts}{Silke Neumann}{}{Deutsch}{Bündisch/Pfadfinderlied}{
\verse{
\li{\Am[]Nordwärts, nordwärts, \G[]woll'n wir ziehen, \Dm[]zu den Bergen \Am[]und den Seen,}
\li{\C[]wollen neues \G[]Land erleben, \Am[]woll'n auf\ch{(} \Dm[]Fahr\Em[)]ten \Am[]geh'n.}
}

\verse{
\li{\Am[]Wollen frei, so \G[]wie ein Vogel, \Dm[]wiegen uns im \Am[]kalten Wind,}
\li{\C[]woll'n den Ruf der \G[]Wildnis hören, \Am[]wenn wir\ch{(} \Dm[]glück\Em[)]lich \Am[]sind.}
}

\verse{
\li{\Am[]Woll'n durch Moor und \G[]Sümpfe waten, \Dm[]abends legen \Am[]uns zur Ruh.}
\li{\C[]Klampfen sollen \G[]leis' erklingen, \Am[]singen\ch{(} \Dm[]im\g{-}\Em[)]mer\g{-}\Am[]zu.}
}

\verse{
\li{\Am[]In der Kohte \G[]brennt ein Feuer, \Dm[]füllt uns alle \Am[]mit Bedacht.}
\li{\C[]Schlaf senkt sich auf \G[]uns hernieder, \Am[]doch die\ch{(} \Dm[]Wild\Em[)]nis \Am[]wacht.}
}

\verse{
\li{\Am[]Käuzchenschreie, \G[]Bäume rauschen \Dm[]bis zum frühen \Am[]Morgengrau.}
\li{\C[]Über ausge\G[]qualmtem Feuer \Am[]strahlt der\ch{(} \Dm[]Him\Em[)]mel \Am[]blau.}
}

\verse{
\li{\Am[]Wenn wir wieder \G[]heimwärts ziehen, \Dm[]sehnet jeder \Am[]sich zurück,}
\li{\C[]denkt an die ver\G[]gangenen Fahrten, \Am[]an ve\ch{(}r\Dm[]gang'\Em[)]nes \Am[]Glück.}
}

\verse{
\li{\Am[]Nordwärts, nordwärts, \G[]woll'n wir wieder, \Dm[]zu den Bergen \Am[]und den Seen,}
\li{\C[]dieses Land noch\G[]mal erleben, \Am[]und auf\ch{(} \Dm[]Fahr\Em[)]ten \Am[]geh'n.}
}
}
\end{document}
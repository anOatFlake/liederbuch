\documentclass[../Liederbuch/LiederbuchGitarristen.tex]{subfiles} 
\begin{document}
\song{L'Estaca}{Lluís Llach}{}{Katalanisch}{}{
\verse{
\li{L'\Em[]avi Si\B[7]set em par\Em[]lava, de bon matí al por\B[7]tal,}
\li{\Em[]Mentre el \B[7]sol esper\Em[]àvem, i els carros \B[7]vèiem pas\Em[]sar.}
\li{Siset, que ne veus l'e\B[7]staca \Em[]on estem tots lli\B[7]gats?}
\li{\Em[]Si no po\B[7]dem des\Em[]fer-nos-en, mai no po\B[7]drem cami\Em[]nar.}
}

\chorus{
\li{\Em[]Si e\B[7]stirem tots, ella cau\Em[]rà,}
\li{i molt de \B[7]temps no pot du\Em[]rar:}
\li{Segur que \Am[]tomba, tomba, \Em[]tomba,}
\li{ben cor\B[7]cada deu ser \Em[]ja.}
\li{Si jo l'es\B[7]tiro fort per \Em[]aquí,}
\li{i tu l'e\B[7]stires fort per \Em[]allà,}
\li{segur que \Am[]tomba, tomba, \Em[]tomba,}
\li{i ens po\B[7]drem allibe\Em[]rar.}
}

\verse{
\li{\Em[]Però, Siset, \B[7]fa molt \Em[]temps ja, les mans se'm van escor\B[7]xant,}
\li{\Em[]i quan la \B[7]força se \Em[]me'n va ella és més \B[7]ampla i més \Em[]gran.}
\li{Ben cert sé que està po\B[7]drida \Em[]però que, Siset, pesa \B[7]tant,}
\li{\Em[]que a cops la \B[7]força m'ob\Em[]lida. Torna'm a \B[7]dir el teu \Em[]cant:}
}

\refrain

\verse{
\li{L'\Em[]avi  Si\B[7]set ja \Em[]no diu res, mal vent que se l'empor\B[7]tà,}
\li{\Em[]ell qui \B[7]sap cap \Em[]a quin indret i jo a \B[7]sota el por\Em[]tal.}
\li{I mentre passen els \B[7]nous vailets, \Em[]estiro el coll per can\B[7]tar}
\li{\Em[]el darrer \B[7]cant d'\Em[]en Siset, el darrer que \B[7]em va ensen\Em[]yar.}
}

\refrain

\outro{
\li{La la la \B[7]la la la la \Em[]la,}
\li{la la la \B[7]la la la la \Em[]la,}
\li{segur que \Am[]tomba, tomba, \Em[]tomba,}
\li{i ens po\B[7]drem allibe\Em[]rar.}
}
\footer{
\dq Der Pfahl\dq{} ist ein katalanisches Kampflied gegen die Franco-Diktatur. Es wurde 1968 von Lluís Llach geschrieben.
Der Pfahl (l'estaca) steht darin sinnbildlich für den Staat (l'estat), den es zu stürzen gilt.
}
}
\end{document}
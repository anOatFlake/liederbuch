\documentclass[../Liederbuch/LiederbuchGitarristen.tex]{subfiles} 
\begin{document}
\song{Ein Hotdog unten am Hafen}{Element of Crime}{}{Deutsch}{Pop}{
\verse{
\li{Ein \Am[]Hotdog unten am \C[]Hafen, vor'm \E[]Einschlafen schnell noch ein \Am[]Bier.}
\li{Dem \C[]Feind einen Tritt in die \Am[]Rippen, und ein paar \G[]Kippen für hinter\C[]her.}
\li{Ein \Am[]Date mit dem Dalai \C[]Lama und ein \E[]Apfelsaft morgens um \Am[]zwei}
\li{und eine \C[]halbautomatische \G[]Waffe ist immer da\C[]bei.}
}

\chorus{
\li{\Am[]Schön, wenn man liebt, was \G[]Mutter Natur einem gibt.}
\li{Was kann \F[]ich dafür, dass \G[]du mich nicht ver\C[]gisst?}
\li{Ein ge\F[]selliges Tier ist das \G[]Schwein und das \C[]Stachelschwein lieber al\F[]lein.}
\li{\C[]Ohne dich will ich nicht, \G[]mit dir kann ich nicht \C[]sein.}
}

\verse{
\li{\Am[]Räucherstäbchen und \C[]Wildreis und \E[]Abende auf dem Bal\Am[]kon.}
\li{In \C[]Eppendorf ist morgen \Am[]Flohmarkt und \G[]jeder nach seiner \C[]Fa\c{c}on.}
\li{Ein \Am[]Date mit dem Dalai \C[]Lama und ein \E[]Griff ins Kosmetikre\Am[]gal}
\li{und wenn's im \C[]Rücken mal weh tut wird \G[]jede Bewegung zur \C[]Qual.}
}

\refrain

\verse{
\li{Eine \Am[]Parkbank in Planten und \C[]Blomen und der \E[]Mond über Alto\Am[]na.}
\li{Ein \C[]Sohn, der bald mal ins \Am[]Bett muss, und \G[]trockene Blumen im \C[]Haar.}
\li{Ein \Am[]Date mit dem Dalai \C[]Lama und ein \E[]Klimpern auf dem Kla\Am[]vier}
\li{und zum \C[]Abschied ein bisschen Ge\G[]fummel hinter der \C[]Tür.}
}

\refrain

\outro{
\li{\C[]Ohne dich will ich nicht, \G[]mit dir kann ich nicht \C[]sein.}
}
\footer{
Das Lied behandelt den deutschen Film \dq Robert Zimmermann wundert sich über die Liebe\dq{} von Leander Haußmann aus dem Jahr 2008. Er entstand nach dem gleichnamigen Roman von Gernot Gricksch, der auch das Drehbuch verfasste. 2007 erhielt der Film den Drehbuchpreis auf den Nordischen Filmtagen Lübeck. Autor Gernot Gricksch erhielt 2008 den Bayerischen Filmpreis für das Beste Drehbuch. Die Filmmusik der Berliner Band Element of Crime wurde 2009 für den Deutschen Filmpreis nominiert
}
}
\end{document}
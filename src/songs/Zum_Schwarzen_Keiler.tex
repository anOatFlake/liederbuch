\documentclass[../Liederbuch/LiederbuchGitarristen.tex]{subfiles} 
\begin{document}
\song{Zum Schwarzen Keiler}{Versengold}{}{Deutsch}{Mittelalterlich}{
\intro{
\chli{\Em \D \Em \D \C \C \B[7] \Em \rep{2}}
}

\verse{
\li{\Em[]In Nienburg an der \D[]Weser steht ein \Em[]altes Fachwerk\D[]haus}
\li{Aus \C[]dem fast alle Tage Frohsinn \B[7]und Gejohle \Em[]klingt}
\li{Da \Em[]schenkt man unent\D[]wegt so manches \Em[]volle Krüglein \D[]aus}
\li{Auf \C[]dass der Weingeist lachend zwischen \B[7]Tischen singt und \Em[]springt}
\li{Auch \C[]Speis und Trank sind kaum ge\B[7]streckt in jenem \Em[]Schankesraum}
\li{Für \C[]ein paar Heller wird er\B[7]füllt dir jeder \Em[]Lastertraum}
}

\chorus{
\li{\C[]Trunkenbolde, \D[]Bauern, Bürger, \B[7]Adelsmänner, \Em[]Steuerwürger}
\li{\C[]Tagediebe, \D[]Beutelschneider,\ch{(} \Em[]Pfeffer\D[)]säcke,\ch{(} \G[]Hunger\B[7)]leider}
\li{\C[]Leichte Weiber, \D[]holde Maiden, \B[7]Pfaffen, Ketzer, \Em[]Christen, Heiden}
\li{\C[]Spielleutzbrut und \D[]Müßigweiler, \B[7]jeder zecht im Schwarzen \Em[]Keiler\D* \Em* \D* \C* \B[7]* \Em}
}

\verse{
\li{\Em[]Der Herr im Hause \D[]gar, er ist ein \Em[]alter Ritters\D[]mann}
\li{\C[]Der hat für die Taverne all sein \B[7]Hab und Gut ver\Em[]prasst}
\li{\Em[]So tauschte er sein \D[]Schwert und Rüstzeug \Em[]gegen Krug und \D[]Kann'}
\li{\C[]Und ist seitdem sich selbst ein allzu\B[7]gern geseh'ner \Em[]Gast}
\li{Man \C[]munkelt gar dass er dort, \B[7]wenn er zuviel \Em[]Branntwein schluckt}
\li{\C[]Ab und an auch mal ein \B[7]Feuer durch den \Em[]Schanksaal spuckt}
}

\refrain

\bridge{
\chli{\Em \D \Em \D \C \C \B[7] \B[7] \rep{2}}
}

\verse{
\li{\Em[]Der Koch verwöhnt den \D[]Gaumen wie es \Em[]kaum ein and'rer \D[]kann}
\li{\C[]Seine Honigrippchen sind die \B[7]feinsten wohl im \Em[]Land}
\li{\Em[]Die Dienerschaft ver\D[]wöhnet jeden \Em[]noch so frommen \D[]Mann}
\li{Und \C[]füllet jeden Gast und seinen \B[7]Becher stets zum \Em[]Rand}
\li{Ob \C[]Bauer oder Edelmann, ein \B[7]jeder \Em[]labt hier gern}
\li{Denn \C[]hier kannst du die Völlerei per\B[7]sönlich \Em[]kennenlern'}
}

\chorus{
\li{\C[]Trunkenbolde, \D[]Bauern, Bürger, \B[7]Adelsmänner, \Em[]Steuerwürger}
\li{\C[]Tagediebe, \D[]Beutelschneider,\ch{(} \Em[]Pfeffer\D[)]säcke,\ch{(} \G[]Hunger\B[7)]leider}
\li{\C[]Schnappesdrosseln, \D[]Schluckerspechte, \B[7]Söldnerschweine, \Em[]Lanzenknechte}
\li{\C[]Heimstattlose \D[]und Soldaten,\ch{(} \Em[]Leichtma\D[)]trosen und\ch{(} \G[]Pi\B[7)]raten}
\li{\C[]Leichte Weiber, \D[]holde Maiden, \B[7]Pfaffen, Ketzer, \Em[]Christen, Heiden}
\li{\C[]Spielleutzbrut und \D[]Müßigweiler, \C[]jeder \B[7]zecht im Schwarzen \Em[]Keiler\D* \Em* \D* \C* \B[7]* \Em}
}
}
\end{document}
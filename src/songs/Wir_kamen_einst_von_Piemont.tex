\documentclass[../Liederbuch/LiederbuchGitarristen.tex]{subfiles}
\begin{document}
\song{Wir kamen einst von Piemont}{trad.}{}{Deutsch}{Mittelalterlich}{
\intro{
\chli{(\C \G[)] (\D \G[)] (\C \G[)] (\D \G[)]}
}
\verse{
\li{\brep Wir \G[]kamen \D[7] einst von Pie\G[]mont - in nicht sehr \C[]glänzen\D[7]der Fa\G[]çon.\erep}
\li{Völlig leer der Magen und der \D[7]Ranzen - völlig durcheinander die Fi\G[]nanzen,}
\li{wir hatten keinen Hel\D[7]ler \G[]mehr.}
}

\chorus{
\li{\brep \C[]Alles durchei\G[]nander, alles \D[7]kreuz und \G[]quer.\erep}
}

\verse{
\li{\brep Und einen \D[7] Hunger hatten \G[]wir - \dq Frau Wirtin, \C[]sagt was \D[7] bietet \G[]ihr?\dq{}\erep}
\li{\dq Wein und Bier und ein Ka\D[7]ninchen - alles durcheinander und ein \G[]Hühnchen,}
\li{Suppe hab ich auch, was wollt \D[7] ihr  \G[]mehr?\dq}
}

\refrain

\verse{
\li{\brep Mal richtig \D[7] schlafen wollen \G[]wir - \dq Frau Wirtin, \C[]sagt was \D[7] bietet \G[]ihr?\dq{}\erep}
\li{\dq Hinten raus ist meine \D[7] Kammer - alles durcheinander, welch ein \G[]Jammer!}
\li{Vorne raus die Zofe, man \D[7] hat's \G[]schwer!\dq}
}

\refrain

\verse{
\li{\brep So gegen \D[7] elf da hörte \G[]man - Frau Wirtin \C[]fing zu \D[7] schimpfen \G[]an.\erep}
\li{\dq Ach mein Scharnier ist ganz ver\D[7]bogen - alles durcheinander, unge\G[]logen,}
\li{seht euch doch vor, ich bitt' \D[7] euch \G[]sehr!\dq}
}

\refrain

\verse{
\li{\brep Und dann nach\D[7]her um Mitter\G[]nacht - da hat's ganz \C[] fürchter\D[7]lich ge\G[]kracht.\erep}
\li{Ein altes Bett zerbrach ganz \D[7] plötzlich - alles durcheinander, wie ent\G[]setzlich!}
\li{Und die kleine Zofe sprach: \dq Das war \D[7] zu \G[]schwer!\dq}
}

\refrain

\verse{
\li{\brep Und kommen \D[7] Sie an diesen \G[]Ort - so grüßen \C[]Sie die \D[7] Wirtin \G[]dort.\erep}
\li{Die zum Schlafen nie allein ins Bett sich \D[7] legte - ihren hübschen Hintern so adrett be\G[]wegte,}
\li{doch die kleine Zofe, die bewegt \D[7] noch \G[]mehr.}
}

\refrain

}
\end{document}
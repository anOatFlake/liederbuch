\documentclass[../Liederbuch/LiederbuchGitarristen.tex]{subfiles} 
\begin{document}
\song{Die Ballade vom roten Haar}{T:Paul Zech M:Peter Rohland}{}{Deutsch}{}{
\verse{
\li{Im \Am[]Sommer war das \Dm[]Gras so tief, daß jeder \G[]Wind daran vorüber\C[]lief.}
\li{Ich habe \Am[]da dein \Em[]Blut ge\G[]spürt, und wie es \F[]heiß zu mir herüber\Am[]rann.}
\li{Du \Am[]hast nur mein Ge\Dm[]sicht berührt, da starb er \G[]einfach hin, der harte \C[]Mann.}
\li{Weil's solche \Am[]Liebe \Em[]nicht mehr \G[]gibt: Ich hab mich \F[]in dein rotes Haar ver\Am[]liebt.}
}

\verse{
\li{Im \Am[]Feld den ganzen \Dm[]Sommer war der rote \G[]Mohn so rot nicht wie dein \C[]Haar.}
\li{Jetzt wird es \Am[]abge\Em[]mäht - das \G[]Gras. Die bunten \F[]Blumen welken auch da\Am[]hin.}
\li{Und \Am[]wenn der rote \Dm[]Mohn so blaß geworden \G[]ist, dann hat es keinen \C[]Sinn,}
\li{dass es noch \Am[]weiße \Em[]Wolken \G[]gibt: Ich hab mich \F[]in dein rotes Haar ver\Am[]liebt.}
}

\verse{
\li{Du \Am[]sagst, dass es bald \Dm[]Kinder gibt, wenn man sich \G[]in dein rotes Haar ver\C[]liebt;}
\li{So rot wie \Am[]Mohn, so \Em[]weiß wie \G[]Schnee, im Herbst, da \F[]kehren viele Wunder \Am[]ein.}
\li{Wa\Am[]rum soll's auch bei \Dm[]uns nicht sein, du bliebst im \G[]Winter doch mein rotes \C[]Reh.}
\li{\brep Und wenn es \Am[]tausend \Em[]Schön're \G[]gibt: Ich hab mich \F[]in dein rotes Haar ver\Am[]liebt.\erep}
}
}
\end{document}
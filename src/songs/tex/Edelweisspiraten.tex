\documentclass[../Liederbuch/LiederbuchGitarristen.tex]{subfiles} 
\begin{document}
\song{Edelweißpiraten}{Hans-Jörg Maucksch/Herwig Steymans}{}{Deutsch}{Bündisch/Pfadfinderlied}{
\verse{
\li{\C[]Sie saßen \G[]oft beim Märchen\F[]see am Lager\C[]feuer,\E}
\li{\Am[]sie wollten \F[]leben, wie es \G[]ihnen ge\C[]fiel.}
\li{\C[]Der neue \G[]Kurs im deutschen \F[]Land war nicht ge\C[]heuer,\E}
\li{\Am[]sie wollten \F[]frei sein mit Ge\G[]sang, Gitarren\C[]spiel.}
\li{\G[]Mit ihrer \F[]Kleidung nahmen \C[]sie's nicht so ge\G[]nau,}
\li{\G[]ganz offen \F[]trugen sie das \C[]Edelweiß zur \G[]Schau und das war \F[]gut, sie hatten \C[]Mut.}
}

\verse{
\li{\C[]Sie hatten \G[]nichts im Sinn mit \F[]braunen Nazi\C[]horden,\E* \Am[]sie hielten \F[]nichts von dem Ge-}
\li{\G[]schrei nach Heil und \C[]Sieg. Was war denn \G[]nur aus ihrem \F[]Vaterland ge\C[]worden?\E}
\li{\Am[]Man schürte \F[]offen den ver\G[]brecherischen \C[]Krieg.}
\li{\G[]Da gab's nur \F[]eins zu tun: Be\C[]frei'n wir dieses \G[]Land,}
\li{\G[]da durfte \F[]keiner ruh'n: Wir \C[]leisten Wider\G[]stand! Sie hatten \F[]Mut und das war \C[]gut.}
}

\verse{
\li{\C[]Da gab's 'nen \G[]Güterzug mit \F[]Kriegsmaschinen und \C[]Waffen\E* \Am[]und was man \F[]sonst noch braucht}
\li{\G[]für einen Völker\C[]mord. Da machten \G[]sie sich an den \F[]Gleisen kurz zu \C[]schaffen,\E* \Am}
\li{der Zug er\F[]reichte nie\G[]mals den Bestimmungs\C[]ort.}
\li{\G[]Und Essens\F[]marken vom \C[]Parteibüro der \G[]Stadt,}
\li{\G[]waren plötzlich \F[]weg und Zwangsar\C[]beiter wurden \G[]satt. Sie hatten \F[]Mut, sie hatten \C[]Mut.}
}

\chorus{
\li{\C[]Vielleicht wird \G[]morgen schon \F[]eine neue\ch{(} \C[]Zeit an\G[)]fangen,}
\li{\C[]vielleicht ist \G[]morgen schon \Dm[]der Spuk vor\Am[]bei.}
\li{\C[]Vielleicht wird \G[]morgen schon \F[]eine neue\ch{(} \C[]Zeit an\G[)]fangen,}
\li{\C[]vielleicht ist \G[]morgen schon \F[]der Spuk vor\C[]bei.}
}

\verse{
\li{\C[]Sie glaubten \G[]fest daran, dass \F[]sie den Sieg er\C[]ringen,\E* \Am[]sie glaubten \F[]fest daran:}
\li{Aus \G[]Schaden wird man \C[]klug. Sie glaubten \G[]fest daran als \F[]sie zum Galgen \C[]gingen.\E* \Am}
\li{Sie glaubten \F[]fest daran als \G[]man sie vorher \C[]schlug.}
\li{\G[]Und diese \F[]Angst, die hinter \C[]jeder Folter \G[]steht,}
\li{\G[]die ist so \F[]groß, dass man den \C[]besten Freund ver\G[]rät. Versteht man \F[]gut, versteht man \C[]gut.}
}

\verse{
\li{\C[]Sie stehen \G[]heute noch auf \F[]manchen schwarzen \C[]Listen.\E* \Am[]Ich möcht' fast \F[]meinen}
\li{heut' ist's \G[]wieder mal so\C[]weit. In Amt und \G[]Würde sitzen \F[]wieder mal Fa\C[]schisten.\E* \Am}
\li{Und zum to\F[]talen Krieg ist \G[]mancher schon \C[]bereit.}
\li{\G[]Nur seh' ich \F[]Tausende - und \C[]das beruhigt mich \G[]sehr}
\li{\G[]die zeigen \F[]offen das zer\C[]brochene Ge\G[]wehr! Denn das macht \F[]Mut, denn das macht \C[]Mut.}
}

\chorus{
\li{\C[]Und dann wird \G[]morgen schon \F[]eine neue\ch{(} \C[]Zeit an\G[)]fangen,}
\li{\C[]Und dann ist \G[]morgen schon \Dm[]der Spuk vor\Am[]bei.}
\li{\C[]Und dann wird \G[]morgen schon \F[]eine neue\ch{(} \C[]Zeit an\G[)]fangen,}
\li{\C[]Und dann ist \G[]morgen schon \F[]der Spuk vor\C[]bei.}
}
\footer{Als Edelweißpiraten werden informelle Gruppen deutscher Jugendlicher mit unangepasstem, teilweise oppositionellem Verhalten im Deutschen Reich von 1939 bis 1945 und während der ersten Nachkriegsjahre bezeichnet. Das Edelweiß war eines unter vielen Kennzeichen der nach 1936 verbotenen Bündischen Jugend.}
}
\end{document}
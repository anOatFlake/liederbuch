\documentclass[../Liederbuch/LiederbuchGitarristen.tex]{subfiles} 
\begin{document}
\song{Wunder der Schwerkraft}{Blutjungs}{}{Deutsch}{Splatterpop}{
\verse{
\li{\G[(III)]Auf einer \C[(III)]Brücke stand ein \D[(V)]Junge, und er \C[]sah mich fröhlich \G[]an.}
\li{Er spuckte \C[]über das Ge\D[]länder, auf die \C[]graue Auto\G[]bahn.}
}

\verse{
\li{\G[]Ich fragte \C[]mich, was er hier \D[]machte, was ihn \C[]wohl hier oben \G[]hielt.}
\li{Interes\C[]siert er sich für \D[]Autos oder \C[]hat er nur ge\G[]spielt?}
}

\verse{
\li{\G[]Er hob den \C[]Finger \D[]und dann zeigte \C[]er}
\li{\G[]auf einen \C[]dunkelroten \D[]Fleck auf dem \C[]Teer. Und er sang:}
}

\chorus{
\li{\G[]Katzen fallen \C[]gar nicht immer \D[]nur auf Ihre \C[]Füßchen.}
\li{\G[]Katzen gehen \C[]manchmal auch ka\D[]putt!\C}
\li{\G[]Katzen fallen \C[]gar nicht immer \D[]nur auf Ihre \C[]Füßchen.}
\li{\G[]Katzen fallen \C[]manchmal auf den \D[]Kopf - wenn man sie \C[]richtig werfen \G[]tut!}
}

\verse{
\li{\G[]Provo\C[]kant war seine \D[]These, und es \C[]schien mir sehr ge\G[]wagt,}
\li{dass ein \C[]Bub von acht, neun \D[]Jahren, gegen \C[]altes Wissen \G[]quakt.}
\li{Aus einem \C[]Körbchen zog \D[]er ein graues \C[]Tier. \G[]Und im Ver\C[]suchsaufbau \D[]bewies er \C[]mir:}
}

\refrain

\verse{
\li{\G[]Nun trag ich \C[]mühsam Deine \D[]Katze, und das \C[]Vieh ist scheiße \G[]schwer.}
\li{Du lässt sie \C[]sonst nicht aus den \D[]Augen, aber \C[]Du bist beim Fri\G[]seur.}
\li{Und während \C[]die dort Deine \D[]Locken nachblon\C[]dieren,}
\li{\G[]werd' ich mal \C[]eben meine \D[]Rückhand austrai\C[]nieren!}
}

\repref{2}
\footer{Was ist dran an der fallenden Katze? Katzen verfügen über einen Reflex, der ihnen hilft, sich im Fall für den Aufprall optimal auszurichten. Der Körperbau der Katze tut ein Übriges, um einen Sturz abzumildern, denn er stellt eine optimale Kombination aus dehnbaren Muskeln und Sehen und stabilen Knochen dar. Das Umdrehen benötigt jedoch Zeit. Bei einem Fall vom Couchtisch kann sich selbst die schnellste Katze nicht mehr umdrehen. 1987 wird im Journal of the American Veterinary Medical Association berichtet, dass von 132 New Yorker Katzen bei einer durchschnittlichen Fallhöhe von 5,5 Stockwerken 90 Prozent den Sturz, wenn auch teilweise schwer verletzt, überlebten. Das siebte Stockwerk scheint für Katzenstürze eine magische Zahl zu sein. Denn während bis dahin die Überlebenschance der stürzenden Katze mit zunehmender Stockwerkzahl sinkt, steigt sie danach wieder. Grund hierfür ist, dass ab einer bestimmten Falldauer der Luftwiderstand die Fallbeschleunigung aufwiegt. Nichtsdestotrotz ist dieses Lied keine Aufforderung, Katzen rumzuwerfen, sondern sollte vielmehr als Parodie auf dieses Phänomen verstanden werden.}
}
\end{document}
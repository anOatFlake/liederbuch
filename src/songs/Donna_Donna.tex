\documentclass[../Liederbuch/LiederbuchGitarristen.tex]{subfiles}
\begin{document}
\song{Donna Donna}{T:A. Kevess/T. Schwartz M:S. Secunda}{}{Englisch}{}{

\verse{
\li{\Am[]On a \E[]wagon \Am[]bound for \E[]market \Am* there's a \Dm[]calf with a \Am[]mournful \E[]eye.}
\li{\Am[]High a\E[]bove him \Am[]there's a \E[]swallow \Am* winging \Dm[]swiftly \Am[]through \E[]the \Am[]sky.}
}

\chorus{
\li{\G* How the winds are \C[]laugh\Am[]ing,  they \G[]laugh with all their \C[]might,}
\li{\G* laugh and laugh the \C[]whole day \Am[]through and \E[]half the summer's \Am[]night.}
\li{\E[]Donna donna donna \Am[]donna, \G[]donna donna donna \C[]don,}
\li{\E[]donna donna donna \Am[]donna, \E[]donna donna donna \Am[]don.}
}

\verse{
\li{\Am[]\dq Stop com\E[]plaining\dq{} \Am[]said the \E[]farmer, \Am*  \dq Who told \Dm[]you a \Am[]calf to \E[]be?}
\li{\Am[]Why don't \E[]you have \Am[]wings to \E[]fly with \Am* like the \Dm[]swallow, so \Am[]proud \E[]and \Am[]free?\dq}
}

\refrain

\verse{
\li{\Am[]Calves are \E[]easily \Am[]bound and \E[]slaughtered, \Am* never \Dm[]knowing the \Am[]reason \E[]why.}
\li{\Am[]But who\E[]ever \Am[]treasures \E[]freedom \Am* like the \Dm[]swallow has \Am[]learned \E[]to \Am[]fly.}
}

\refrain

\footer{Donna Donna handelt von einem Kalb, dass sich nicht wehren kann und steht so als Bild für die Juden im Dritten Reich. In dieser Zeit wurde das Lied für das Musical Esterke geschrieben. Den Text verfasste Secunda ursprünglich auf jiddisch, später auch auf englisch. Aber erst die Überarbeitung von Schwartz verbreitete sich. Bis heute gibt es eine Vielzahl von Interpretationen.}
}
\end{document}
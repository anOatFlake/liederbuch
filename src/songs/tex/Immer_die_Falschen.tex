\documentclass[../Liederbuch/LiederbuchGitarristen.tex]{subfiles} 
\begin{document}
\song{Immer die Falschen}{Blutjungs}{}{Deutsch}{Splatterpop}{
\verse{
\li{\sch{(}\D[)] Guten Abend, Herr Großmann\G[], Sie sind schon lange da\C[]bei,}
\li{haben Sie einen Tipp \Em für die jungen Wilden aus der Rock'n'Roll-\C[]Nach\G[]wuchs\C[]da\D[]tei?}
\li{\NC Unter uns, Herr Großmann\G[], gibt's da nicht einen \C Rat?}
\li{Der der Jugend den \Em Weg ebnen kann, ham Sie nicht einen \C Spruch \G auf \C der \D Naht?}
\li{Eins kann ich \A sagen, wenn man mich lässt, die ganz große \C Wahrheit, mein Mani\D[]fest:}
}

\chorus{
\li{\D Es sind immer die \G Falschen, \D die das Hemd aus\Em[]zieh'n,}
\li{\C es sind immer die \G Feisten, \D mit dem Doppel\Em[]kinn,}
\li{\C die schlecht Täto\G[]wierten \D mit dem blauen Del\Em[]fin\D,}
\li{\C es sind immer die \G Falschen, \D die das Hemd aus\G[]zieh'n.}
}

\verse{
\li{\D Spaß beiseite, Herr Großmann, \G Sie wissen doch \C auch,}
\li{wie schwer es für die Jung\Em[]kapellen ist, so viele Schwätzer, so \C viel \G Schall \C und \D Rauch.}
\li{\NC Darum frag' ich, Herr Großmann\G[], nochmal mit Be\C[]dacht:}
\li{Haben Sie nicht eine \Em Lebensweisheit, die den Kleinen den \C Weg \G leich\C[]ter \D macht?}
\li{Nun, selbstvers\A[]tändlich helf' ich da gern, es gibt eine \C Wahrheit, doch die will keiner \D hör'n:}
}

\chorus{
\li{\D Es sind immer die \G Falschen, \D die das Hemd aus\Em[]zieh'n,}
\li{\C es sind immer die \G Feisten, \D mit dem Doppel\Em[]kinn,}
\li{\C die heftig Be\G[]haarten \D mit Kontaktaller\Em[]gien\D,}
\li{\C es sind immer die \G Falschen, \D die das Hemd aus\G[]zieh'n.}
}

\chorus{
\li{Es sind immer die \G Falschen, \D die das Hemd aus\Em[]zieh'n,}
\li{\C die mit den Ek\G[]zemen \D und dem Bauch an den \Em Knien.}
\li{\C Die schlampig Ge\G[]piercten \D mit dem Nippel\Em[]ring\D,}
\li{\C es sind immer die \G Falschen, \D die das Hemd aus\G[]zieh'n.}
}
}
\end{document}
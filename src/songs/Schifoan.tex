\documentclass[../Liederbuch/LiederbuchGitarristen.tex]{subfiles} 
\begin{document}
\song{Schifoan}{Wolfgang Ambros}{}{Deutsch}{Pop}{
\intro{
\chli{\G \Em \C \D \G \Em \C \D}
}

\verse{
\li{Am \G[]Freitog auf'd \Em[]Nocht mon\C[]tier' i die \D[]Schi}
\li{\G[]auf mei' \Em[]Auto und \C[]dann begeb' i \D[]mi}
\li{in's \G[]Stubai\Em[]tal oder noch \C[]Zell am \D[]See,}
\li{weil \G[]durt auf die \Em[]Berg ob'm ham's \C[]immer an \D[]leiwand'n \G[]Schnee.\Em* \C}
}

\chorus{
\li{\D[]Weil i wü', \G[]Schi\Em[]foan, \ch{Am}Schi\C[]foan, wow wow wow,}
\li{\G[]Schi\Em[]foan, weil \C[]Schifoan is des \Em[]Leiwandste,}
\li{\ch{Am}wos ma sich nur \D[]vurstelln \G[]kann.}
}

\verse{
\li{In der \G[]Fruah bin i der \Em[]Erste der \C[]wos aufe\D[]foart,}
\li{da\G[]mit i ned so \Em[]long auf's \C[]aufefoarn \D[]woart.}
\li{\G[]Ob'm auf der \Em[]Hütt'n kauf' i \C[]ma an Jager\D[]tee,}
\li{weil \G[]so a \Em[]Tee mocht' den \C[]Schnee erst so \D[]richtig \G[]schee.\Em* \C}
}

\refrain

\bridge{
\li{\G[]Und wann der Schnee staubt \Em[]und wann die Sunn' scheint,}
\li{\C[]dann hob' i ollas \D[]Glück in mir vereint.}
\li{\G[]I steh' am Gipfel schau' \Em[]obe ins Tal.}
\li{A \C[]jeder is glücklich, a \D[]jeder fühlt sich wohl, und wü nur}
}

\refrain

\verse{
\li{Am \G[]Sonntag auf'd \Em[]Nacht mon\C[]tier' i die \D[]Schi}
\li{\G[]auf mei' \Em[]Auto, aber \C[]dann überkommt's \D[]mi}
\li{und i \G[]schau' no amoi \Em[]aufe und \C[]denk' ma \dq aber \D[]wo\dq.}
\li{I \G[]foar' no ned z'\Em[]Haus i bleib' am \C[]Montog \D[]a no \G[]do.\Em* \C}
}

\refrain
\footer{Ambros besingt das 'Schifoan' im Stubaital und in Zell am See als 'des Leiwandste, wos ma sich nur vurstelln kann' (standarddt.: das Schönste, das man sich nur vorstellen kann). Das Lied gilt als österreichische Wintersporthymne.}
}
\end{document}
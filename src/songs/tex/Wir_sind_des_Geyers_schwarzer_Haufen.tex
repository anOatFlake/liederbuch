\documentclass[../Liederbuch/LiederbuchGitarristen.tex]{subfiles} 
\begin{document}
\song{Wir sind des Geyers schwarzer Haufen}{T:Heinrich von Reder M:Fritz Sotke}{}{Deutsch}{Mittelalterlich}{
\verse{
\li{Wir \Dm[]sind des Geyers Schwarzer Haufen, \Gm[]heyah hey\Dm[]oh}
\li{Wir \Gm[]wollen mit Pfaff und \Dm[]Adel raufen, \C[]heyah hey\A[]ohohohoh}
}

\chorus{
\li{\brep \Dm[]Spieß voran, hey!, \Gm[]drauf und \Dm[]dran. \C[]Setzt aufs Klosterdach den \Dm[]ro - \A[]ten \Dm[]Hahn\erep}
}

\verse{
\li{Jetzt \Dm[]geht es Schloss, Abtei und Stift, \Gm[]heyah hey\Dm[]oh}
\li{Uns \Gm[]gilt nichts als die \Dm[]Heil'ge Schrift, \C[]heyah hey\A[]ohohohoh}
}

\refrain

\verse{
\li{Als \Dm[]Adam grub und Eva spann, \Gm[]heyah hey\Dm[]oh}
\li{Wo \Gm[]war denn da der \Dm[]Edelmann? \C[]Heyah hey\A[]ohohohoh}
}

\refrain

\verse{
\li{Wir \Dm[]wollen's Gott im Himmel klagen, \Gm[]heyah hey\Dm[]oh}
\li{Dass \Gm[]wir die Pfaffen nicht \Dm[]dürfen totschlagen, \C[]heyah hey\A[]ohohohoh}
}

\refrain

\verse{
\li{Wir \Dm[]woll'n nicht länger sein ein Knecht, \Gm[]heyah hey\Dm[]oh}
\li{leib\Gm[]eigen, fröhnig, \Dm[]ohne Recht, \C[]heyah hey\A[]ohohohoh}
}

\refrain

\verse{
\li{Ge\Dm[]schlagen gehen wir nach Haus, \Gm[]heyah hey\Dm[]oh}
\li{Die \Gm[]Enkel fechten's \Dm[]besser aus, \C[]heyah hey\A[]ohohohoh}
}

\refrain
\footer{Der Schwarze Haufen war ein Odenwälder Bauernheer während des Deutschen Bauernkrieges (1524-1526) unter Führung von Florian Geyer von Giebelstadt. Die Taten des Schwarzen Haufens und insbesondere des Florian Geyer wurden im Laufe der Zeit zunehmend verherrlicht und besonders während der deutschen Romantik regelrecht glorifiziert. In diesem Zusammenhang ist das Lied \dq Wir sind des Geyers schwarzer Haufen\dq{} (Text: Entstanden nach dem 1. Weltkrieg in Kreisen der Jugendbewegung unter Verwendung von Textteilen des Gedichtes 'Ich bin der arme Kunrad' von Heinrich von Reder (1885), Melodie: Fritz Sotke (1919)) zu sehen, das an die Forderungen und die Rhetorik der Bauern des 16. Jahrhunderts angelehnt ist.}
}
\end{document}